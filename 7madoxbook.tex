\documentclass{book}
\usepackage[utf8]{inputenc}
\usepackage{fancybox,framed}
\usepackage[framemethod=tikz]{mdframed}
\usepackage{amsmath}
\usepackage[most]{tcolorbox}

\newtcolorbox{Ex}[3][]{
arc=5mm,
lower separated=false,
fonttitle=\bfseries,
colbacktitle=black!10,
coltitle=grey!50!black,
enhanced,
attach boxed title to top left={xshift=0.5cm,
        yshift=-2mm},
colframe=grey!50!black,
colback=grey!10,
overlay={
\node[draw=grey!50!black,thick,
%inner sep=2mm,
fill= green!10,rounded corners=1mm, 
yshift=0pt, 
xshift=-0.5cm, 
left, 
text=grey!50!black,
anchor=east,
font=\bfseries] 
at (frame.north east) {#3};},
title=#2 \thetcbcounter,#1}

\newtcolorbox{Définition}[3][]{
arc=5mm,
lower separated=false,
fonttitle=\bfseries,
colbacktitle=green!10,
coltitle=green!50!black,
enhanced,
attach boxed title to top left={xshift=0.5cm,
        yshift=-2mm},
colframe=green!50!black,
colback=green!10,
overlay={
\node[draw=green!50!black,thick,
%inner sep=2mm,
fill= green!10,rounded corners=1mm, 
yshift=0pt, 
xshift=-0.5cm, 
left, 
text=green!50!black,
anchor=east,
font=\bfseries] 
at (frame.north east) {#3};},
title=#2 \thetcbcounter,#1}

\newtcolorbox{Théorème}[3][]{
arc=5mm,
lower separated=false,
fonttitle=\bfseries,
colbacktitle=purple!10,
coltitle=purple!50!black,
enhanced,
attach boxed title to top left={xshift=0.5cm,
        yshift=-2mm},
colframe=purple!50!black,
colback=green!10,
overlay={
\node[draw=purple!50!black,thick,
%inner sep=2mm,
fill= green!10,rounded corners=1mm, 
yshift=0pt, 
xshift=-0.5cm, 
left, 
text=purple!50!black,
anchor=east,
font=\bfseries] 
at (frame.north east) {#3};},
title=#2 \thetcbcounter,#1}

\newtcolorbox{Propriété}[3][]{
arc=5mm,
lower separated=false,
fonttitle=\bfseries,
colbacktitle=red!10,
coltitle=red!50!black,
enhanced,
attach boxed title to top left={xshift=0.5cm,
        yshift=-2mm},
colframe=red!50!black,
colback=green!10,
overlay={
\node[draw=red!50!black,thick,
%inner sep=2mm,
fill= red!10,rounded corners=1mm, 
yshift=0pt, 
xshift=-0.5cm, 
left, 
text=red!50!black,
anchor=east,
font=\bfseries] 
at (frame.north east) {#3};},
title=#2 \thetcbcounter,#1}
\newtcolorbox{HP}[3][]{
arc=5mm,
lower separated=false,
fonttitle=\bfseries,
colbacktitle=yellow!10,
coltitle=yellow!50!black,
enhanced,
attach boxed title to top left={xshift=0.5cm,
        yshift=-2mm},
colframe=yellow!50!black,
colback=yellow!10,
overlay={
\node[draw=green!50!black,thick,
%inner sep=2mm,
fill= yellow!10,rounded corners=1mm, 
yshift=0pt, 
xshift=-0.5cm, 
left, 
text=yellow!50!black,
anchor=east,
font=\bfseries] 
at (frame.north east) {#3};},
title=#2 \thetcbcounter,#1}
\title{Algébre Linéaire}
\usepackage{framed}
\begin{document}
\maketitle{7madox}
\tableofcontents
\chapter{Les Espaces Vectoriels}
\section{Introduction}
Un espace vectoriel est une structure algébrique stable par addition interne (de vecteurs) et par multiplication externe (par un scalaire).
\subsection{Définitions:}
\begin{Définition}[]{Définition}{}
Soit \(E\) un ensemble non vide, et \((K,+,x)\) un corps dont le neutre pour la loi "\(+\)" est noté, et pour la loi "\(x\)" est noté .
\newline
On note l’ensemble E muni d’une loi interne  \("+"\)   et d’une loi externe  \("."\).
\newline
On dit que  est un K-espace vectoriel lorsque :
\begin{itemize}
    \item[i)] \((E,+)\) forme un groupe abélien, dont l'élément neutre, noté \(0_E\) , est appelé le vecteur nul.
    \item[ii)] La loi est distributive par rapport à la loi + : 

\( \forall \lambda \in K, \forall (x, y) \in E^2 , \lambda.(x+y) =(\lambda.x) + (\lambda.y)\)
    \item[iii)] \(\forall (\lambda , \mu) \in K^2, \forall x \in E, (\lambda+\mu).x=(\lambda.x)+(\mu.x) et (\lambda*\mu).x=\lambda.(\mu.x)\)
    \item[iv)] \(\forall x \in E, 1_K.x=x\)
\end{itemize}

Les éléments de E s'appellent des \textbf{vecteurs} et les éléments de \(K\) des \textbf{scalaires}.
\end{Définition}
\subsection{Exemples: }
\((R^2,+,*) \) est un \(R\)-espace vectoriel , en effet:
\begin{itemize}
    \item (i): (\(R^2, +)\) est un groupe abélien de neutre (0,0).
    \item (ii): Soient \(\lambda \in R, (x,y) \in R^2 * R^2\) tel que \(x=(x_1, x_2)\) et \(y= (y_1,y_2)\), on a :
    \begin{center}
    \begin{align*}
    \lambda(x+y)=
    \lambda(x_1+y_1,x_2+y_2) \\
    \Leftrightarrow \qquad &=(\lambda x_1+\lambda y_1,\lambda x_2+\lambda y_2) \\
    \Leftrightarrow \qquad &=((\lambda x_1,\lambda x_2) +(\lambda y_1,\lambda y_2))\\
\Leftrightarrow \qquad &=\lambda(x_1,x_2)+\lambda(y_1,y_2)\\
    \Leftrightarrow \qquad &=(\lambda.x)+(\lambda.y)
    \end{align*}
    \end{center}
    \item (iii): Soient \((\lambda, \mu) \in R^2, x \in R^2\) tel que \(x=(x_1, x_2)\)  on a:
\begin{center}
\begin{align*}
    (\lambda + \mu).x =(\lambda+\mu).(x_1,x_2)
\\
    \Leftrightarrow  &=(\lambda.x_1+\mu.x_1,\lambda.x_2+\mu.x_2) \\
    \Leftrightarrow  &=(\lambda.x_1,\lambda.x_2)+(\mu.x_1, \mu.x_2)\\
    \Leftrightarrow  &=(\lambda.x)+(\mu.x)
\\
(\lambda * \mu).x=(\lambda *\mu._x_1,\lambda*\mu.x_2) \\
    \Leftrightarrow  &=\lambda.(\mu.x_1, \mu.x2)
\end{align*}
(Car la multiplication est associative dans \(R\).)

\item (iv) Soit \(x \in R^2\) tel que \(x=(x_1,x_2)\) on a:

\(1_R.x=1.x=(1.x_1,1.x_2)=(x_1,x_2)=x\).
(Car \(x_1, x_2\) sont dans R.)
    
\end{center} 
\end{itemize}
Donc \((R^2,+,.)\) est un R-espace vectoriel, on peut visualiser cet espace et illustrer les proposition et les theorèmes qu'on va étudier sur cet espace , on peut également les illustrer à travers le R-espace vectoriel \((R^3, +,.) \) pour lequel la démonstration est similaire à ce qu'on a déja fait.
\paragraph{Illustration (vecteur, addition , par scalaire)}
\subsection{Propriétés:}
\begin{Propriété}[]{Propriétés}{}
\begin{itemize}
    \item \(\forall \lambda \in K, \forall x \in E, \lambda.x = 0 \Leftrightarrow \lambda=0_K \) ou \(x=0_E.\)
    \item \(\forall x \in E, (-1_K).x=-x.\) (\(-1_K\) est l'opposé de \(1_K\) dans K et -x est l'opposé de x dans E.)
\end{itemize}

\end{Propriété}
\paragraph{Remarque: }
On verra lors de l'étude de l'algebre linéaire plusieurs exemples d'espaces vectoriels dont on va détaillera l'étude et les propriétés prochainement.

\section{Sous-espace vectoriel}
\subsection{Définition:}
\begin{Définition}[]{Définition}{}
Soit E un K-espace vectoriel et F une partie de E, F est un sous-espace vectoriel si la restriction des lois "+", "." sur F lui confère la structure d'un espace vectoriel , c'est à dire , si F est aussi un K-espace vectoriel.
\end{Définition}
\paragraph{Exemples: }
Si E est un K-espace vectoriel alors E et \(\{0_E\}\) sont les deux des sous espaces vectoriels de E.
\subsection{Caracterisation:}
\begin{Propriété}[]{Caractérisation: }
Soit F une partie de E. On peut montrer que F est un sous-espace vectoriel  de E si les conditions suivantes sont réalisées:

i)\(F \neq \varnothing. \)

ii)\(\forall (x,y) \in E, \forall (\lambda, \mu) \in K², \lambda.x+\mu.y \in F. \)
\end{Propriété}
\paragraph{Remarque:}
\(0_E\) est dans F et \(0_F=0_E\), généralement , on montre qu'un ensemble est un sous-espace vectoriel d'un K espace vectoriel en utilisant cette caractérisation en commencant par montrer que \(0_E\) est dans F, souvent on montre aussi qu'un ensemble est un K espace vectoriel en montrant qu'il est en effet un sous-espace vectoriel d'un K espace vectoriel usuel.
\paragraph{Exemples: }
\subsection{Sous-espace vectoriel engendré}

\subsubsubsection{Combinaison linéaire}
 
Soit I un ensemble eventuellement infini.
\begin{Définition}[]{Définition}{}
On appelle combinaison linéaire d'éléments de la famille de vecteurs ${\displaystyle (x_{i})_{i\in I}}$ tout vecteur v de ${\displaystyle E}$ tel qu’il existe une famille de scalaires ${\displaystyle (\lambda _{i})_{i\in I}\in K}  $
tel que les ${\displaystyle (\lambda _{i})_{i\in I}}$ sont nuls sauf un nombre fini d'entre eux et qu'elle vérifie ${\displaystyle v=\sum _{i\in I}\lambda _{i}x_{i}}.$
\end{Définition}
\paragraph{Exemples:-illus }
Soient \((0,1), (3,2), (5,1)\) trois vecteurs du \(\R\)-espace vectoriel \(R^2\), alors (-1,-3/5) est une combinaison linéaire de ces vecteur, en effet on a: 

\((1,-1,2/5) \in R\) tels que \((-1,-3/5) = 1*(0,1)-1*(-3,2) + 2/5*(5,1)\) 
\subsubsubsection{S-ev engendré par une famille de vecteurs}
\begin{Définition}[]{Définition}{}
On appelle sous-espace vectoriel de ${\displaystyle E}$ engendré par la famille ${\displaystyle (x_{i})_{i\in I}}$, qu'on note ${\displaystyle \operatorname {Vect} \left((x_{i})_{i\in I}\right)}$: l’ensemble des combinaisons linéaires d'éléments de 
${\displaystyle (x_{i})_{i\in I}}{\displaystyle (x_{i})_{i\in I}}$ :
\begin{framed}
${\displaystyle \operatorname {Vect} \left((x_{i})_{i\in I}\right)=\left\{\left.\sum _{i\in I}\lambda _{i}x_{i}~\right|~(\forall i \in I), \lambda _{i}\in K\right\}}$.
\end{framed}
L'ensemble ${\displaystyle \operatorname{Vect} \left((x_{i})_{i\in I}\right)}$ est, au sens de l'inclusion, le plus petit sous-espace vectoriel de ${\displaystyle E}$ contenant tous les ${\displaystyle x_{i}}.$
\end{Définition}
\paragraph{Exemples: -illus}
\subsubsubsection{S-ev engendré par une partie}
\begin{Définition}[]{Définition}{}
Soient ${\displaystyle E}$ un espace vectoriel sur ${\displaystyle K}$ et ${\displaystyle A\subset E}.$

On appelle sous-espace vectoriel engendré par ${\displaystyle A} $, et l'on note ${\displaystyle \operatorname {Vect} (A)}$l’ensemble de toutes les combinaisons linéaires d'éléments de ${\displaystyle A}$ :
\begin{framed}
${\displaystyle \operatorname {Vect} (A)=\left\{\left.\sum _{i=1}^{n}\lambda _{i}.a_{i}~\right|~n\in \mathbb {N} ,(\lambda _{1},\cdots ,\lambda _{n})\in K^{n},(a_{1},\cdots ,a_{n})\in A^{n}\right\}}.$
\end{framed}
Il s'agit, au sens de l'inclusion, du plus petit espace vectoriel contenant ${\displaystyle A}.$
\end{Définition}
\paragraph{Exemples-illus}
\subsection{Somme de Sous-espaces vectoriels}
\begin{Définition}[]{Définition}{}
Soient ${\displaystyle F} et {\displaystyle G}$ deux sous-espaces vectoriels de ${\displaystyle E}.$ On définit la somme de ${\displaystyle F} $et ${\displaystyle G}$ comme l’ensemble :
\begin{framed}
${\displaystyle F+G=\{u+v\mid (u,v)\in F\times G\}}.$
\end{framed}
\end{Définition}
\paragraph{Remarque:} ${\displaystyle F+G=\operatorname {Vect} \left(F\cup G\right)}.$
\paragraph{Exemples-illus}
\subsection{Intersubsection de Sous-espaces vectoriels}
\begin{Propriété}[]{Proposition}{}
Toute intersubsection de sous-espaces vectoriels de ${\displaystyle E}$ est un sous-espace vectoriel de ${\displaystyle E}$.
\end{Propriété}
\paragraph{Démonstration: }
Soient E un K-espace vectoriel, F et G deux sous espaces vectoriel de E,
\\
\begin{itemize}
    \item[i)] puisque F et G sont deux sous espaces vectoriel de E , \(0_{E} \in F\) et \(0_{E} \in G\) donc \(0_{E} \in F \cap G\) donc \(F \cap G \neq \varnothing\).
    \item[ii)] Soient \((x,y,\lambda) \in F \cap G * F \cap G * K\), on a \((x,y,\lambda) \in F  * F  * K\) donc \(x+\lambda y \in F\) car F est un sous espace vectoriel de E,
    \\ De même \((x,y,\lambda) \in  G *  G * K\) donc \(x+\lambda y \in G\) car G est un sous espace vectoriel de E,
    \\ donc \(x+\lambda y \in F \cap G\).
\end{itemize}
Donc d'après la caractérisation des sous espaces vectoriel on déduit que \(F \cap G\) est un sous espace vectoriel de E.
\newline
\\ Pour une intersubsection de plus de deux sous espaces vectoriel , on montre la proposition en utilisant la récurrence sur le nombre des sous espaces vectoriel, en fait l'hérédite se montre exactement comme on a fait ci-dessus.
\paragraph{Remarque:}
Cette dernière n'est pas toujours vraie pour la réunion.
\begin{HP}[]{Proposition-HP}{}
Si K est un corps commutatif et E un K-espace vectoriel et F et G deux sous espaces vectoriel de E tel que \(F \cup G\) est un sous espace vectoriel de E, alors \(F \subset G\) ou \(G \subset F\).
\end{HP}
\paragraph{Démonstration:}
Raisonnement par absurde:
\newline
\\ Soient K est un corps commutatif et E un K-espace vectoriel et F et G deux sous espaces vectoriel de E tel que \(F \cup G\) est un sous espace vectoriel de E,
\newline
\\ Supposons le contraire de " \(F \subset G\) ou \(G \subset F\) ", c'est à dire "\(F \not\subset G\) et \(G \not\subset F\)" alors il existe \(x \in F, x \notin G\)
 et \(y \in G, y \notin F\). 
 \newline
 On a \(x+y \in F \cup G\) donc \( x+y \in F\) ou \(x+y \in G\)
 Si \( x+y \in F\) , puisque \(x \in F, -x \in F\), donc \(x+y+(-x)=x+y-x=y \in F\), ce qui est absurde.
 De même si \( x+y \in G\), on obtient \(x \in G\), ce qui est aussi absurde.
 \newline
 \\ Donc dans tous les cas , par raisonemment par absurde on obtient \(F \subset G\) ou \(G \subset F\).

\subsection{Supplémentaire de Sous-espaces vectoriels}
\begin{Définition}[]{Définition}{}
Deux sous-espaces vectoriels ${\displaystyle F}$ et ${\displaystyle G}$ de ${\displaystyle E}$ sont dits supplémentaires si ${\displaystyle F+G=E}$ et ${\displaystyle F\cap G=\{0\}}$ autrement dit : si

${\displaystyle E\subset F+G\quad {\text{et}}\quad F\cap G\subset \{0\}}$

On note alors ${\displaystyle E=F\oplus G}.$
\end{Définition}
\paragraph{Exemples-illus}
\subsection{Produit Cartésien de Sous-espaces vectoriels}
\begin{Définition}[]{Définition}{}
Soient ${\displaystyle E}$ et ${\displaystyle F}$ deux ${\displaystyle K}$-espaces vectoriels. 
\newline
On définit l'espace produit de ${\displaystyle E}$ et ${\displaystyle F}$ comme l'ensemble produit ${\displaystyle E\times F}$, muni des deux lois suivantes, qui en font un ${\displaystyle \mathbb{K}}$-espace vectoriel :

${\displaystyle (x,y)+_{E\times F}(x',y'):=(x+_{E}x',y+_{F}y')\quad {\text{et}}\quad \lambda \cdot _{E\times F}(x,y):=(\lambda \cdot _{E}x,\lambda \cdot _{F}y)}$
\end{Définition}
\paragraph{Exemples:}
\section{Famille de Vecteurs}
\subsection{Famille génératrice}
\begin{Définition}[]{Définition}{}
La famille ${\displaystyle (x_{i})_{i\in I}} $est dite génératrice de ${\displaystyle E}$ si
${\displaystyle E=\mathrm {Vect} \left((x_{i})_{i\in I}\right)}$.
Cela équivaut à dire que tout vecteur de ${\displaystyle E}$ s'exprime  comme combinaison linéaire de la famille ${\displaystyle (x_{i})_{i\in I}}$:
\begin{framed}
${\displaystyle \forall v\in E\quad \exists (\lambda _{i})_{I}\in K^{Card(I)} \quad v=\sum \lambda _{i}x_{i}}$
\end{framed}
\end{Définition}
\paragraph{Remarque:}

Si A est une partie de E et ${\displaystyle E=\mathrm {Vect} \left(A\right)}$, on dit que A est une partie génératrice de E.

\paragraph{Exemples: }
\begin{itemize}
    \item fonctions.
\end{itemize}
\subsection{Famille libre, liée}
Définitions:
\begin{Définition}[]{Définitions}{}
On dit que la famille ${\displaystyle (x_{i})_{i\in I}}$ est libre, ou que les vecteurs ${\displaystyle x_{i}}$ sont linéairement indépendants, si aucun vecteur n'est combinaison linéaire des autres vecteurs.

Cela équivaut à dire que :
\begin{framed}
${\displaystyle \forall (\lambda _{i})_{i\in I}\in K^{Card(I)}\quad \sum \lambda _{i}x_{i}=0\Rightarrow \forall i\in I\quad \lambda _{i}=0}.$
\end{framed}

Une famille qui n'est pas libre est dite liée. Elle est donc liée si un vecteur est une combinaison linéaire des autres dans , c'est-à-dire :
\begin{framed}
${\displaystyle \exists (\lambda _{i})_{i\in I}\in K^{Card(I)}\quad \sum \lambda _{i}x_{i}=0{\text{ et }}(\lambda _{i})_{i\in I}\neq (0)}.$
\end{framed}
\end{Définition}
\paragraph{Exemples : (3 classiques de gourdon) avec dem}
Dans le R-espace vectoriel des fonctions continues de R dans R , les familles suivantes sont des familles libres:
\begin{itemize}
    \item[i)] \((f_{\lambda})_{\lambda \in R} \) oû \(f_{\lambda}: R  \rightarrow R \left x \mapsto \exp{(\lambda x)} \).
    \item[ii)] \((f_{\lambda})_{\lambda \in R} \) oû \(f_{\lambda}: R  \rightarrow R \left x \mapsto \cos{(\lambda x)} \).
    \item[iii)] \((f_{\lambda})_{\lambda \in R} \) oû \(f_{\lambda}:  R \rightarrow R \left x \mapsto |x-\lambda| \).
    \item[iv)] \((f_{k})_{ \in \mathbb{N} \) oû \(f_{n}: R  \rightarrow R \left x \mapsto \cos{(x^n)} \).
\end{itemize}
Soit  \(\mathbb{K}\) un sous corps de $\mathbb{C}$ , dans le \(\mathbb{K}\)-espace vectoriel  \(K[X]\), \\ la famille \((1,X,...,X^{n-p-1},P(X),P(X+1),...,P(X+p))\) avec \(P \in K[X]\) de degré \(n \geq 1\) et \(p \in [|0,n|]\), est libre quelque soit \(p \in [|0,n|]\).
\paragraph{Démonstration:}
\begin{itemize}
    \item[i)] On note le \(\mathbb{R}\]-espace vectoriel par E et on montre par absurde que la famille est libre.
    \\ On suppose que la famille \((f_{\lambda})_{\lambda \in R} \) oû \(f_{\lambda}: R  \rightarrow R \left x \mapsto \exp{(\lambda x)} \) est liée. \\
    On aura donc:
    \\ \\\(\exists (\lambda_{i})_{1 \leq i \leq n} \in \mathbb{R}^n, \exists (\mu_{i})_{1 \leq i \leq n} \in \mathbb{R}^n\), avec les \(\mu_{i}\) sont non tous nuls  telles que \begin{center}\(\sum_{1 \leq i \leq n} \mu_{i} \lambda_{i} = 0_{E}\)
    \end{center}
    \\ \\ On indexe les \(\lambda_{i}\) de sorte que \(\lambda_{i} \le \lambda_{j}\) si \(j \le i\), c'est à dire: \(\lambda_{n} \le ... \le \lambda_{1}\).  
    \\ \\ Soit \(k \in [|1,n|]\) tel que \(k=min\{\{1,..,n\}|\mu_{i} \neq 0_{\mathbb{R}}\}\) 
    \\ \\ \(Lim_{x \rightarrow +\infty} exp(-\lambda_{1} x)\sum_{i=k}^n \mu_{i}exp(\lambda_{i} x)=\sum_{i=k}^n \mu_{i}exp(\lambda{i}-\lambda_{1})=\mu_{1}\).
    \\ car pour tout \(i \geq 2, \lambda_{i}-\lambda_{1} \le 0.\).\\
    Or \(\sum_{i=k}^n\mu{i}f_{\lambda_{i}}=0,\), donc \(u_{1}=0\), ce qui est absurde.
    \\ \begin{framed}On conclut par raisonnement par absurde que la famille \((f_{\lambda})_{\lambda \in R} \) oû \(f_{\lambda}: R  \rightarrow R \left x \mapsto \exp{(\lambda x)} \) est libre.
    \end{framed}
    
    \item[ii)] La famille  \((f_{\lambda})_{\lambda \in R} \) oû \(f_{\lambda}: R  \rightarrow R \left x \mapsto \cos{(\lambda x)} \) étant libre est équivalent à 
    \begin{center}
    \begin{framed}
    \(\forall (\mu_{i}) \in K^{Card(I)} \sum_{i\in I}\mu_{i}x_{i}=0 \Rightarrow \forall \in I, \mu_{i}=0.\)
    \end{framed}
    \end{center}
    \\ l'ensemble I qui indexe les \(\mu_{i}\) contient qu'un nombre fini d'éléments non nuls qu'on note \(n \in \mathbb{N}^*\)
    donc \((\mu_{i})_{i\in I}=(\mu_{i})_{i\in [|1,n|]}\), 
    \\ On conclut donc qu montrer que la liberté de la famille est équivalent à montrer que :
    \begin{center}
        \begin{framed}
            \(P: \forall n \in \mathbb{N}^*, \forall (\lambda_{i})_{i\in [|1,n|]} \in\mathbb{R}^n\)  \(\forall (\mu_{i})_{i\in[|1,n|]} \in \mathbb{R}^n \sum_{i=1}^n \mu_{i}f_{\lambda_{i}}=0_{E}\)  \(\Rightarrow \forall i \in [|1,n|], \mu_{i}=0_{\mathbb{R}}\)
        \end{framed}
    \end{center}
    Or le \(cos\) étant paire on va prendre (les \(\lambda_{i}\) distincts dans \(\mathbb{R}^+\)).
    On montrera alors cette proposition par récurrence simple sur \(n \in \mathbb{N}^*\)
    \\ \textbf{Initialisation: P(1)}
    \\ Pour n=1, soient \((\lambda_{1},\mu_{1}) \in \mathbb{R}^2\) telles que \(\mu_{1}f_{\lambda_{1}}=0_{E}\), on a donc \(\ \forall x \in R, \mu_{1}.\cos{\lambda_{1}x}=0_{R}\), pour \(x=0, cos(0)=1\) donc \(\mu_{1}=0_{R})\).
    \\ On a montré donc \begin{center}
        \begin{framed}
            \(\forall \mu_{1} \in \mathbb{R},\forall \lambda_{1} \in \mathbb{R}^+, (\mu_{1}f_{\lambda_{1}}=0_{E})\Rightarrow (\mu_{1}=0_{R})\).
        \end{framed}
    \end{center} 
    \textbf{Hérédité: \(P(n) \Rightarrow P(n+1)\)}
    \\ Soit \(n \geq 1\), supposant la proposition vraie au rang n, et montrons la au rang n+1:
    \\ Soient \((\mu_{i})_{i \in [|1,n+1|]}\) et \((\lambda_{i})_{i \in [|1,n+1|]}\) telles que \\
\(\sum_{i \in [|1,n+1|]}\mu_{i}f_{\lambda_{i}}=0_{E}\)
\\ Par double dérivation et puisque (les \(\lambda_{i}\) distincts dans \(\mathbb{R}^+\)) on a: \\ \(\sum_{i=1}^{n+1}\mu_{i}(-\lambda_{i}^2)f_{\lambda_{i}}=0_{E}\) (1)
\\ et par multiplication par \(\lambda_{n}^2\) : \(\sum_{i=1}^{n+1}\mu_{i}(\lambda_{n}^2)f_{\lambda_{i}}=0_{E}\) (2)
\\ On ajoute (1) et (2) et on obtient \(\lambda_{n+1}^2\) : \(\sum_{i=1}^{n}\mu_{i}(\lambda_{n+1}^2-\lambda_{i}^2)f_{\lambda_{i}}=0_{E}\)
\\ D'après l'hypothèse de la récurrence "P(n)", on a : \\ \(\forall i \in [|1,n|], \mu_{i}(\lambda_{n+1}^2-\lambda{i}^2)=0_{R}\)
\\ Et puisque les \(\lambda_{i}\) distincts dans \(\mathbb{R}^+\) on déduit que : \\ \(\forall i\in [|1,n|], \mu_{i}=0_{R}\)
\\ Donc \(\sum_{i \in [|1,n+1|] } \mu_{i}f_{\lambda_{i}}=0_{E}\) devient \(\mu_{n+1}f_{\lambda_{n+1}}=0_{E}\)
\\ ce qui veut dire \(\forall x \in R,\mu_{n+1}f_{\lambda_{n+1}}(x)=0_{R} \) donc \(\mu_{n+1}=0_{R}\).
\\ Donc \( \forall i \in [|1,n+1|], \mu_{i}=0_{R}\)
\\ \\ On a montré que , en supposant que P(n) est vraie on déduit que : \(\sum_{i \in [|1,n+1|]}\mu_{i}f_{\lambda_{i}}=0_{E} \Rightarrow \forall i \in [|1,n+1|],\mu_{i}=0_{R}\)

\\ Finalement par raisonnement par récurrence : 
\\\begin{framed}
La famille
\((f_{\lambda})_{\lambda \in R} \) oû \(f_{\lambda}: R  \rightarrow R \left x \mapsto \cos{(\lambda x)} \) est libre.
\end{framed}
    \item[iii)] On montrerai par absurde que la famille \((f_{\lambda})_{\lambda \in R} \) oû \(f_{\lambda}:  R \rightarrow R \left x \mapsto |x-\lambda| \) est libre.
    \\ Supposant que la famille \((f_{\lambda})_{\lambda \in R} \) est liée, alors comme précédemment la libérté de la famille étant équivalent à \\ \(P: \forall n \in \mathbb{N}^*, \forall (\lambda_{i})_{i\in [|1,n|]} \in\mathbb{R}^n\)  \(\forall (\mu_{i})_{i\in[|1,n|]} \in \mathbb{R}^n \sum_{i=1}^n \mu_{i}f_{\lambda_{i}}=0_{E}\)  \(\Rightarrow \forall i \in [|1,n|], \mu_{i}=0_{\mathbb{R}}\) \\
    On supposera : \\ \(P(bar):\exists n \in \mathbb{N}^*, \exists (\lambda_{i})_{i\in [|1,n|]} \in \mathbb{R}^n ,\exists (\mu_{i})_{i\in[|1,n|]} \in \mathbb{R}^n \sum_{i=1}^n \mu_{i}f_{\lambda_{i}}=0_{E}\)  et  les \(\mu_{i}\) non tous nuls.
    \\ D'après la proposition il existe \(\lambda_{0} \in \mathbb{R}\) tel que \(f_{\lambda_{0}\) est combinaison linéaire des \((f_{\lambda})_{\lambda \in \mathbb{R}-\{\lambda_{0}\}\) ce qui est équivalent à :
    \\ \(\exists n \in \mathbb{N}^*, \exists (\lambda_{i})_{i\in [|1,n|]} \in \mathbb{R}^n-\{\lambda_{0}\} ,\exists (\mu_{i})_{i\in[|1,n|]} \in \mathbb{R}^n f_{\lambda_{0}}=\sum_{i=1}^n \mu_{i}f_{\lambda_{i}}\)
    \\ Pour tout \(\lambda \in \mathbb{R}\) la fonction de \(\mathbb{R}\) dans \(\mathbb{R}\) \(x \mapsto \left|x-\lambda|\right\) est dérivable en tout \(x\neq \lambda\) donc \(\forall i \in [|1,n|], f_{\lambda_{i}}\) est dérivable sur \(\lambda_{0}\) car \(\forall i \in [|1,n|], \lambda_{0} \neq \lambda_{i}\)
    \\ Cependant, par addition de fonction dérivable au même point et multiplication par scalaires \(\sum_{i=1}^n \mu_{i}f_{\lambda_{i}}\) est dérivable en \(\lambda_{0}\) or \(f_{\lambda_{0}}=\sum_{i=1}^n \mu_{i}f_{\lambda_{i}}\) donc \(f_{\lambda_{0}}\) est dérivable en \(\lambda_{0}\) ce qui est absurde.
    \\ Finalement on a montré que la proposition \(P: \forall n \in \mathbb{N}^*, \forall (\lambda_{i})_{i\in [|1,n|]} \in\mathbb{R}^n\)  \(\forall (\mu_{i})_{i\in[|1,n|]} \in \mathbb{R}^n \sum_{i=1}^n \mu_{i}f_{\lambda_{i}}=0_{E}\)  \(\Rightarrow \forall i \in [|1,n|], \mu_{i}=0_{\mathbb{R}}\) est vraie, donc \begin{framed}
        La famille \((f_{\lambda})_{\lambda \in R} \) oû \(f_{\lambda}:  R \rightarrow R \left x \mapsto |x-\lambda| \) est libre.
    \end{framed}
    \item[iv)]
\end{itemize}
\subsection{Base d'un espace vectoriel}
\begin{Définition}[]{Définition}{}
Une famille ${\displaystyle (e_{i})_{i\in I}}$ de vecteurs de ${\displaystyle E}$ est une base de ${\displaystyle E}$ si et seulement si elle est libre et génératrice de ${\displaystyle E}$.

Ce qui est équivalent à:

Tout vecteur de ${\displaystyle E}$ s'écrit comme une combinaison linéaire unique des ${\displaystyle e_{i}}$ :
\begin{framed}
${\displaystyle \forall v\in E\quad \exists !(\lambda _{i})_{i\in I}\in K^{Card(I)}\quad v=\sum \lambda _{i}e_{i}}$
\end{framed}
Les ${\displaystyle (\lambda _{i})_{i\in I}}$ sont tous nuls sauf un nombre fini, et sont alors appelées les coordonnées de ${\displaystyle v}$ dans la base ${\displaystyle (e_{i})_{i\in I}}$.
\end{Définition}
\paragraph{Exemple:}
\begin{itemize}
    \item (1,i) est une base du R-espace vectoriel \((C,+,.)\).
    \item il existe une certaine type de base dit priviligée qui s'appele base canonique, elle apparait comme la base la plus simple pour un espace vectoriel:
    Soit \(n \in \N^*\), la base canonique du R-espace vectoriel \(\R^n\) est \(B_c=(e_1,e_2,\dots, e_n\) avec 
    \(\forall i \in [|1,n|], e_i=(0,\dots,1,\dots,0)\) , le 1 étant le i-ème coefficient du vecteur \(e_i\), tout simplement \(e_i\) est le vecteur dont tous ces coefficients sont nuls sauf le i-ème qui égale à 1.

    Cette base parait celle la plus naturelle à considérer, en fait prenant l'exemple de \((R^3,+,.)\) un vecteur de ce dernier s'écrit sous la forme : \((a,b,c)\) avec \(a,b,c \in R\) donc :

    \((a,b,c)=a*(1,0,0)+b*(0,1,0)+c*(0,0,1)\) 

    du coup la base canonique de \((R^2,+,.)\) est \(((1,0,0),(0,1,0),(0,0,1)), e_1=(1,0,0), e_2=(0,1,0), e_3=(0,0,1)\),  il faut bien retenir et comprendre la base canonique car elle est utilisée extensivement en algébre linéaire.
\end{itemize}

\begin{Théorème}[]{Théorème}{}
Si E est un espace vectoriel engendré par une famille de vecteurs finie ${\displaystyle (x_{i})_{i\in [|1,n|]}, n \in N^*}$, alors:
\begin{itemize}
    \item[i)] Toute famille libre a au plus n vecteurs.
    \item[ii)] Toute famille génératrice a au moins n vecteurs.
\end{itemize}

\end{Théorème}
\paragraph{Démonstration}
\section{Dimension}
\begin{Définition}[]{Définition}{}
Un espace vectoriel est dit de dimension finie s'il existe un famille génératrice finie de celui ci, dans le cas contraire , on parle d'espace vectoriel de dimension infinie.
\end{Définition}
\paragraph{Remarque: }
Il faut faire attention au corps de base de l'espace vectoriel car il agit sur la dimension , par exemple \((\C,.,+\) est un espace vectoriel de dimension 2 si on le voit en tant que R-espace vectoriel, en fait sa base est \((1,i)\) , cependant en tant qu'un C-espace vectoriel , il est de dimension 1.

On aborde cette remarque avec plus de détail dans les subsections concernant la notion hors programme d'extension de corps.
\paragraph{Exemples: }
\begin{itemize}
    \item Soit \(n \in \N^*\) alors \((R^n,+,*)\) est de dimension n.
    
\end{itemize}
\subsection{Théorème d'existence de base:}
\begin{Théorème}[]{Théorème d'existence de base}{}
Soit E un espace de dimension finie, si ${\displaystyle G=(x_{i})_{i\in I}}$ est une famille génératrice de E et il existe ${\displaystyle{J \in I}}$pour laquelle la famille  ${\displaystyle L=(x_{i})_{i\in J}}$ est libre, alors il existe une base B de E tq ${\displaystyle {L\subset B\subset G}}$ 
\end{Théorème}
\paragraph{Démonstration: }
\subsection{Théorème de la base incomplète:}
\begin{Théorème}[]{Théorème de la base incomplète}{}
Soit E un K-espace vectoriel de dimension finie,

De toute famille génératrice de E, on peut en extraire une base en prenant les vecteurs linéairement indépendants.

Toute famille libre de E peut être complétée en une base, en ajoutant des vecteurs qui ne sont pas une combinaison linéaire des vecteurs de la famille libre.
\end{Théorème}

\begin{Définition}[]{Définition}{}
Soit E un K-espace vectoriel de dimension finie, 
toutes les bases de E ont le même cardinal qu'on note ${\displaystyle{dim_{K}E.}}$

Par convention : ${\displaystyle{dim_{K} E=0 $ pour $E=\{0\}.}}$
\end{Définition}
\subsection{Proposition:}
\begin{Propriété}[]{Proposition}{}
Soit E un K-espace vectoriel de dimension ${\displaystyle{n\in \N^*}}$:

Toute famille de n vecteurs libre de est une base de E.

Toute famille de n vecteurs génératrice est une base de E.
\end{Propriété}
\paragraph{Démonstration: }
\subsection{Proposition:}
\begin{Propriété}[]{Proposition}{}
Soit E un K-espace vectoriel de dimension finie et F un sous espace vectoriel de E alors:

${\displaystyle{dim_{K} E \leq dim_{K} F}}$.

Cas d'égalité:
Si ${\displaystyle{dim_{K} E = dim_{K} F}}$ , on a ${\displaystyle{E = F}}$.
\end{Propriété}
\paragraph{Démonstration: }
\subsection{Formule de Grassman:}
\begin{Propriété}[]{Formule de Grassman}{}
Soit E un K-espace vectoriel et $E_1$ et $E_2$ deux sous espace vectoriels de E alors $ E_1+E_2$ est aussi un sous espace vectoriel de E et on a la formule suivante:
\begin{framed}
${\displaystyle{dim_{K} E_1  dim_{K} E_2= dim_{K} (E_1+E_2)-dim_{K}(E_1 \cap E_2)}}$
\end{framed}
\begin{framed}
${\displaystyle{dim_{K} E = dim_{K} E_1 + dim_{K} E_2}}$ et $E_1 \cap E_2 = \{0\}.$
\newline
$\iff dim_{K} E = dim_{K} E_1 + dim_{K} E_2$ et $ E=E_1+E_2 \iff E=E_1\oplus E_2 $
\end{framed}
\end{Propriété}
\chapter{Les Applications Linéaires}
\section{Definitions}
Soient E et F deux K-espaces vectoriels:
\subsection{Application Linéaire:}
\subsubsubsection{Définition: }
\begin{Définition}[]{Définition}{}
Une application ${\displaystyle u:E\to F}$ est dite linéaire si elle vérifie :
\newline
\begin{itemize}
    \item[i)] L'additivité: $\forall (x, y) \in E^2 ,  u(x+y)=u(x)+u(y)$
    \item[ii)] L'homogéniété: $ \forall x \in E,\space  \forall \lambda \in K, \space  u(\lambda x)=\lambda u(x)$
\end{itemize}

ou encore, si elle vérifie :
\begin{center}
    ${\displaystyle \forall (x,y) \in E^2, \forall \lambda \in K , u(\lambda x+y)=\lambda u(x)+u(y)}$.
\end{center}
Elles s'appellent donc des homomorphismes (ou tout simplement morphismes) et leur ensemble est un K-espace vectoriel noté $\(L(E,F)\)$
\end{Définition}


Exemples: 
\begin{Propriété}[]{Propriétés}{}
\begin{itemize}
    \item L'addition, composée, de deux applications linéaires est une application linéaire.
    \item Une application linéaire reste linéaire si elle est multipliée par un scalaire.
    \item La reciproque d'une bijection linéaire est encore linéaire.
\end{itemize}
\end{Propriété}
\subsubsubsection{Termes: }
\begin{Propriété}[]{Propriétés}{}
On appelle :
\begin{itemize}
    \item Endomorphisme de E : toute application linéaire de E dans E.
    \item Isomorphisme de E vers F:  toute bijection linéaire de E dans F ;
    \item Automorphisme de E: tout endomorphisme bijectif de E, ou encore, tout isomorphisme de E dans E.
    \item Forme linéaire sur E:  toute application linéaire de E dans K.
\end{itemize}
\end{Propriété}
\paragraph{Remarque: }
\begin{itemize}
    \item L'ensemble \(L(E, E)\) des endomorphismes de E se note plus simplement \(L(E)\).
    \item L'ensemble des automorphismes de E s’appelle le groupe linéaire de E et se note \(GL(E)\).
    \item L'ensemble \(L(E, K)\) des formes linéaires sur E se note plus simplement E* et porte le nom de dual de E. (On va voir plus tard).
\end{itemize}

\subsection{Noyau, Image: }
\subsubsubsection{Définitions: }
\begin{Définition}[]{Définitions}{}
Soit ${\displaystyle u\in \operatorname {L} (E,F)}$. On appelle :
\begin{itemize}
    \item L'ensemble $f(E)= {\displaystyle \{u(x)\mid x\in E\}}$, s'appele l'image de u et est noté ${\displaystyle Im(u).}$
    \item L'ensemble $f^{-1}(\{0_{F}\}) = {\displaystyle \{x\in E\mid u(x)=0_{F}\}}$, s'appele le noyau de u et est noté ${\displaystyle Ker(u)}$.
\end{itemize}
\end{Définition}
\paragraph{Remarque: }
La notation "Ker" vient du mot allemand "Kern" qui signifie noyau.
\subsubsubsection{Théorème}
\begin{Théorème}[]{Théorème}{}
Soit ${\displaystyle u\in \operatorname {L} (E,F)}$.

L'image réciproque par u d'un sous-espace vectoriel de F est un sous-espace vectoriel de E ;
L'image directe par u d'un sous-espace vectoriel de E est un sous-espace vectoriel de F.
\end{Théorème}
\subsubsubsection{Corollaire}
\begin{Propriété}[]{Corollaire}{}
\begin{itemize}
    \item Ker(u) est un sous-espace vectoriel de E.
    \item Im(u) est un sous-espace vectoriel de F.
\end{itemize}
\end{Propriété}
\subsubsubsection{Théorème: }
\begin{Théorème}[]{Théorème}{}
Soit ${\displaystyle u\in \operatorname {L} (E,F)}$.

\begin{itemize}
    \item  u est injective si et seulement si Ker(u) = {0}.
    \item u est surjective si et seulement si Im(u) = F.
\end{itemize}
\end{Théorème}
\paragraph{Demonstration: }
\section{Caracterisation par les bases: }
\subsection{Théorème: }
\begin{Théorème}[]{Théorème}{}
Pour toute base ${\displaystyle (e_{i})_{i\in I}} de {\displaystyle E}$, l'application 

\begin{align*}
    ${\displaystyle \operatorname {L} (E,F)\to F^{Card(I)}}$
    
    $u\mapsto \left(u(e_{i})\right)_{i\in I}}$
\end{align*}

est bijective.
\end{Théorème}
\subsection{Théorème: }
\begin{Théorème}[]{Théorème}{}
Soit ${\displaystyle u\in \operatorname {L}}$ (E,F).
\begin{itemize}
    \item ${\displaystyle u}$ est surjective si et seulement si l'image par ${\displaystyle u}$ d'au moins une famille génératrice de ${\displaystyle E}$ est génératrice de ${\displaystyle F}$ (de plus, l'image par ${\displaystyle u}$ de toute famille qui engendre ${\displaystyle E}$ est alors génératrice de ${\displaystyle F}$).
    \item ${\displaystyle u}$ est injective si et seulement si l'image par ${\displaystyle u}$ d'au moins une base de ${\displaystyle E}$ est libre (de plus, l'image par ${\displaystyle u}$ de toute famille libre est alors libre) ;
${\displaystyle u}$ est un isomorphisme si et seulement si l'image par ${\displaystyle u}$ d'au moins une base (ou de toute base) de ${\displaystyle E}$ est une base de ${\displaystyle F}$.
\end{itemize}  
\end{Théorème}

\section{Projections, Symétries: }
\subsection{Projecteurs: }
\subsubsubsection{Définition}
\begin{Définition}[]{Définition}{}
Soient $E_{1}$ et $E_{2}$ deux sous espaces vectoriels de E tel que ${\displaystyle E_{1} \oplus E_{2}=E}$ ie:
${\displaystyle \forall x \in E, \exists! (x_{1}, x_{2}) \in E_{1} {x} E_{2}}$ tel que ${\displaystyle x=x_{1}+x_{2}}$

L'application ${\displaystyle  p:E\to E}$
${x\mapsto x_{1}}$ s'appelle la projection sur $E_{1}$ parallèlement à $E_{2}$.
\begin{itemize}
    \item[i)] ${\displaystyle p \in L(E)}.$
    \item[ii)] ${\displaystyle Im(p)=E_{1}}$ et ${\displaystyle Ker(p)=E_{2}}.$
    \item[iii)] ${\displaystyle p \circ p = p}.$
\end{itemize}
Reciproquement si ${\displaystyle p \in L(E)}$ et ${\displaystyle p \circ p = p}$ alors p est un projecteur.
\end{Définition}
\subsubsubsection{Théorème:}
\begin{Théorème}[]{Théorème}{}
    Soit ${\displaystyle p \in L(E)}$
    p est un projecteur ${\Leftrightarrow}$ p est la projection sur Im(p) parallèlement à Ker(p).
    
    Dans ce cas ${\displaystyle E= Im(p) \oplus Ker(p)}$
\end{Théorème}
\subsection{Symétrie:}
\subsubsubsection{Définition:}
\begin{Définition}[]{Définition}{}
    Soient $E_{1}$ et $E_{2}$ deux sous espaces vectoriels de E tel que ${\displaystyle E_{1} \oplus E_{2}=E}$ ie:
${\displaystyle \forall x \in E, \exists! (x_{1}, x_{2}) \in E_{1} {x} E_{2}}$ tel que ${\displaystyle x=x_{1}+x_{2}}$

L'application ${\displaystyle  s:E\to E}$
${x\mapsto x_{1} - x_{2}}$ s'appelle symétrie par rapport à $E_{1}$ parallèlement à $E_{2}$.
\begin{itemize}
    \item[i)] ${\displaystyle s \in L(E)}$.
    \item[ii)] Si ${\displaystyle p \in L(E)}$ est la projection sur $E_{1}$ parallèlement à $E_{2}$ alors: 
${\displaystyle  s=2p-Id_{E}}$
\end{itemize}
\end{Définition}
\subsubsubsection{Proposition: }
\begin{Propriété}[]{Proposition}{}
Dans le cadre du programme , K=$R$ ou $C$ donc on a le résultat suivant:

$s \in L(E)$ est une symétrie $\Leftrightarrow$ $s \circ s=Id_{E}$

Dans ce cas si $p=1/2(s+Id_{E})$, p est un projecteur et s est la symétrie par rapport à Im(p) parallèlement à Ker(p).

On expliquera plus loin, plus ce résultat et d'ou il vient (Voir Notion Caractéristique d'un corps).
\end{Propriété}
\section{L'espace L(E,F):}
\subsection{Théorème: }
\begin{Théorème}[]{Théorème}{}
Si ${\displaystyle E}$ est de dimension finie alors

${\displaystyle \dim \left(\operatorname {L} (E,F)\right)=\dim(E)\times \dim(F).}$.
\end{Théorème}
\paragraph{Remarque: }
Si ${\displaystyle u\in \operatorname {L} (E,F)}$et ${\displaystyle v\in \operatorname {L} (F,G)}$ alors ${\displaystyle v\circ u\in \operatorname {L} (E,G)}$.
\section{Rang:}
\subsection{Définition:}
\begin{Définition}[]{Définition}{}
Le rang d'une application linéaire est la dimension de son image. 
Si ${\displaystyle u:E\to F}$ est une application linéaire alors on note son rang par $rg(u)$ et on a :
    ${\displaystyle rg(u) = dim_{K}(Im(u))}.$
\end{Définition}
\subsection{Théorème:}
\begin{Théorème}[]{Théorème}{}
La composition par un isomorphisme laisse le rang invariant, c'est à dire :
Soit ${\displaystyle u\in \operatorname {L} (E,F)}$:
    
${\displaystyle \forall v\in \operatorname {L} (F,G)}$ bijective ${\displaystyle \mathrm {rg} (v\circ u)=\mathrm {rg} (u)}.$

${\displaystyle \forall v\in \operatorname {L} (F,G)}$ bijective $ {\displaystyle\mathrm {rg} (v\circ u)=\mathrm {rg} (u)}.$
\end{Théorème}
\subsection{Théorème du rang:}
\begin{Théorème}[]{Théorème du rang}{}
    Soit E un K-espace vectoriel de dimension finie et F un espace vectoriel , et ${\displaystyle u \in L(E,F)}$ alors:
    u est de rang fini et on a :
    
    ${\displaystyle dim_{K}(E)=dim_{K}(Im(u))+dim_{K}(Ker(u))}$
    
\end{Théorème}
\begin{Propriété}[]{Corollaire}{}
 Soient E, F deux K-espaces vectoriel de même dimension finie \(u \in L(E,F)\) , alors les assertions suivantes sont équivalentes:
 \\ \begin{itemize}
     \item[i)] u bijective.
     \item[ii)] u surjective.
     \item[iii)] u injective.
 \end{itemize}
\end{Propriété}
\section{Stabilité:}
\begin{Définition}[]{Définition}{}

\end{Définition}
\begin{Propriété}[]{Proposition}{}

\end{Propriété}
\begin{Propriété}[]{Proposition}{}

\end{Propriété}
\section{Exercices:}
\subsection{Projecteurs:}
\subsection{Lemmes de factorisation:}
\begin{Ex}[]{Exercice:}{}
Soient E,F,G 3 K-espaces vectoriels de dimension finie, et soit \(g:E \Rightarrow G\) une application linéaire.
\\ \begin{itemize}
    \item[1)] Soit \(f:E \Rightarrow F\) une application linéaire, montrer que:
    \\ (\( \exists h: F\Rightarrow G \in L(F,G)\)  tel que \(g=h \circ f\).) \(\Leftrightarrow\) (\(Ker f \subset Ker g.\))
    \item[2)] Soit \(h:F \Rightarrow G\) une application linéaire,
    montrer que: 
    \\ (\( \exists f: E\Rightarrow F \in L(E,F)\),  tel que \(g=h \circ f\).) \(\Leftrightarrow\) (\(Im g \subset Im h.\))
    \item[3)] On suppose maintenant que \(g:E \Rightarrow F \in L(E,F)\), montrer que:
    \\ (\(rg g \leq rg f\).) \(\Leftrightarrow\) (\(\exists h \in GL(F)\) et \(k \in L(E)\) tels que \(h \circ g = f \circ k\).)
\end{itemize}
\end{Ex}
\paragraph{Correction:}
\subsection{Inégalité de Sylvester:}
\begin{Ex}[]{Exercice:}{}
Soient E un K-espace vectoriel de dimension finie, F un K-espace vectoriel et \(f,g \in L(E,F)\):
\\ \begin{itemize}
    \item[1)] Montrer que \(|rg f- rg g| \leq rg(f+g) \leq rg f + rg g\).
    \item[2)] Supposant maintenant que f et g sont les deux des endomorphismes de E, montrer que :
    (\(rg(f+g)=rg(f)+rg(g).\)) \(\Leftrightarrow\) (\(Im(f) \cap Im(g)={\varnothing} et Ker(f)+Ker(g)=E\).)
    \item[3)] Montrer l'inégalité de Sylvester : 
    \\ \(rg(f)+rg(g)-dim_{K}(E) \leq rg(fg) \leq min(rg(f),rg(g))\). 
\end{itemize}
\end{Ex}
\paragraph{Correction:}
\subsection{Endomrophismes particuliers: }
\begin{Ex}[]{Exercice:}{}
Soient E un K-espace vectoriel de dimension finie et \(u \in L(E)\):
\\ \begin{itemize}
    \item[1)] Montrer que les assertions sont équivalentes:
    \begin{itemize}
        \item[i)] \(E=Ker u \oplus Im u\).
        \item[ii)] \(\exists v \in L(E), v\circ u =0 et v+u \in GL(E)\).
        \item[iii)] \(Ker u = Ker u^2\).
        \item[iv)] \(Im u = Im u^2\).
        \end{itemize}
\end{itemize}
\end{Ex}
\paragraph{Correction:}


\section{Compléments:}
\subsection{Drapeaux:}
\subsection{Espace vectoriel quotient:}
\begin{Définition}[]{Définitions}{}
Soient E un K-espace vectoriel et F un sous espace vectoriel de E,
\\
La relation R() définit une relation d'equivalence sur E.
L'espace quotient E/F muni des lois "+":\(x+y=x+y\), ".":\(\lambda.x=\lambda.x\)
est un K-espace vectoriel
\\
si E/F est de dimension finie, On appele codimension de F la dimension de E/f tel que :
\\ \(dim_{K}(E/F)=codim_{E}(F)\).
\\ Dans ce cas on dit que F est de codimension finie.
\end{Définition}
\begin{Propriété}[]{Proposition}{}
Soit E un K-espace vectoriel et F un sous espace vectoriel de E.
\\ (F est de codimension finie) \(\Leftrightarrow\) (F admet un supplémentaire S dans E).
\\ Dans ce cas \(dim_{K}(S)=codim_{E}(F)\).
\end{Propriété}
\paragraph{Démonstration:}
j
\begin{Propriété}[]{Corollaire}{}
Si E est un K-espace vectoriel de dimension finie et F un sous espace vectoriel de E , alors F est de codimension finie et :
 \\ \(dim_{K}(E/F)=dim_{K}(E)-dim_{K}(F)\).
 \end{Propriété}
\begin{Propriété}[]{Corollaire}{}
Si E, F sont deux K-espaces vectoriel et \(u \in L(E,F)\) , alors Im(u) est isomorphe à E/Ker(u). 
 \end{Propriété}
\subsection{L'espace L(E):}
\begin{Théorème}[]{Théorème-Programme}{}
    ${\displaystyle (\operatorname {L} (E),+,\cdot ,\circ )}$ est une ${\displaystyle K}$-algèbre associative unifère (non commutative si ${\displaystyle \dim(E)\geq 2}$).
\end{Théorème}
\begin{Définition}[]{Définition}{}
On appelle homothétie \(u \in L(E)\) de rapport \(\lambda \in K\) l'endomorphisme \(\lambda.Id_{E}\)
\end{Définition}
\begin{Propriété}[]{Proposition}{}
    Soit \(u \in L(E)\), 
    \\ (u est une homothétie) \Leftrightarrow (\(\forall x \in E, \) la famille \((x,f(x))\) est liée.)
\end{Propriété}
\paragraph{Démonstration: }
On a 
\begin{Propriété}[]{Proposition-HP}{}
    Le centre du groupe linéare \(Gl(E)\) est l'ensemble des homotéthie de rapport non nul.
\end{Propriété}
\paragraph{Démonstration:}
\begin{Propriété}[]{Proposition-HP}{}
Soit \(E\) un \(\mathbb{K}\)-espace vectoriel de dimension finie \(n \in \mathbb{N}^*\) \(u \in \mathbb{L}(E)\),
    (\(u\) est une homothétie) \(\Leftrightarrow\) (si \(k \in [|1,n-1|]\) u stabilise tous les sous-espace vectoriels de E de dimension k)
\end{Propriété}
\paragraph{Démonstration:}
\subsubsubsection{Idéaux de \(\mathbb{L}(E)\)}
\begin{Propriété}[]{Proposition-HP}{}

\end{Propriété}
\begin{Propriété}[]{Proposition-HP}{}

\end{Propriété}
\begin{Propriété}[]{Proposition-HP}{}

\end{Propriété}
\chapter{Les Matrices}
\section{Généralités: }
\subsection{Définition: }
\begin{Définition}[]{Définition}{}
Une matrice à coefficients dans K est une famille ${\displaystyle \left(a_{i,j}\right)_{(i,j)\in \left[\![1,m]\!\right]\times \left[\![1,n]\!\right]}}$ d'éléments de K.

Les nombres m et n sont appelés dimensions de la matrice. On dit qu'une matrice est de taille m × n.

Les éléments ${\displaystyle a_{i,j}}$ sont appelés coefficients de la matrice.

Une matrice ${\displaystyle A=\left(a_{i,j}\right)_{(i,j)\in \left[\![1,m]\!\right]\times \left[\![1,n]\!\right]}} $est notée de la manière suivante :


${\displaystyle A=
\begin{pmatrix}
a_{1,1} && a_{1,2} && a_{1,3} && \dots && a_{1,n}
\\a_{2,1} && a_{2,2} && a_{2,3} &&\dots && a_{2,n}
\\ \vdots &&\vdots &&\vdots &&\ddots && \vdots &&\vdots&&\vdots&&\vdots
&& && \vdots
\\a_{m,1} && a_{m,2} && a_{m,3}&&\dots && a_{m,n}
\end{pmatrix}}$
\end{Définition}
\paragraph{Remarques: }
\begin{itemize}
\item L’ensemble des matrices de dimensions données à coefficients dans K est noté ${\displaystyle \mathrm {M} _{m,n}\left(K\right)}$.

\item L'ensemble ${\displaystyle \mathrm {M} _{n,n}\left(K\right)}$ est noté plus simplement ${\displaystyle \mathrm {M} _{n}\left(K\right)}$.
    \item Une matrice de largeur n = 1 est appelée vecteur, ou plus spécifiquement vecteur colonne.
\item Une matrice de hauteur m = 1 est appelée vecteur ligne.
\item Une matrice telle que m = n est appelée matrice carrée.
\end{itemize}
\subsection{Opérations sur les matrices: }
\subsubsubsection{Addition: }
\begin{Définition}[]{Définition}{}
    Soient ${\displaystyle A}$ et ${\displaystyle B}$ deux matrices de même taille à coefficients dans K. Alors il est possible de les additionner. Leur somme est une matrice ${\displaystyle A+B}$ à coefficients dans K, de même taille que ${\displaystyle A}$ et ${\displaystyle B}$ :

${\displaystyle \left(a_{i,j}\right)_{(i,j)\in \left[\![1,m]\!\right]\times \left[\![1,n]\!\right]}+\left(b_{i,j}\right)_{(i,j)\in \left[\![1,m]\!\right]\times \left[\![1,n]\!\right]}:=\left(a_{i,j}+b_{i,j}\right)_{(i,j)\in \left[\![1,m]\!\right]\times \left[\![1,n]\!\right]}}.$
\end{Définition}
\subsubsubsection{Produit matricielle:}
\begin{Définition}[]{Définition}{Par un scalaire}
Le produit d'une matrice ${\displaystyle A\in \mathrm {M} _{m,n}(K)}$ par un scalaire ${\displaystyle \lambda \in K}$ est la matrice ${\displaystyle \lambda A\in \mathrm {M} _{m,n}(K)} $dont les coefficients sont ceux de ${\displaystyle A}$ multipliés par ${\displaystyle \lambda }:$
\begin{center}
${\displaystyle \lambda \;\left(a_{i,j}\right)_{(i,j)\in \left[\![1,m]\!\right]\times \left[\![1,n]\!\right]}:=\left(\lambda a_{i,j}\right)_{(i,j)\in \left[\![1,m]\!\right]\times \left[\![1,n]\!\right]}}.$
\end{center}
\end{Définition}
\begin{Définition}[]{Définition}{Produit de deux matrices}
Soient ${\displaystyle A=\left(a_{i,j}\right)_{(i,j)\in \left[\![1,m]\!\right]\times \left[\![1,n]\!\right]}\in \mathrm {M} _{m,n(K)}}$ et ${\displaystyle B=\left(b_{i,j}\right)_{(i,j)\in \left[\![1,n]\!\right]\times \left[\![1,p]\!\right]}\in \mathrm {M} _{n,p(K)}}$ deux matrices telles que . Le produit de ${\displaystyle A}$ par ${\displaystyle B}$ est la matrice suivante :
\begin{center}
${\displaystyle AB=\left(\sum _{k=1}^{n}a_{i,k}b_{k,j}\right)_{(i,j)\in \left[\![1,m]\!\right]\times \left[\![1,p]\!\right]}\in \mathrm {M} _{m,p(K)}}.$
\end{center}
\end{Définition}
\paragraph{Remarque:}
La condition pour la possibilité du produit matricielle est que le nombre de colonnes de la première matrice est égale au nombre de lignes de la deuxième. 
\subsection{Matrices élémentaires: }



\subsubsubsection{Matrice nulle: }
\begin{Définition}[]{Définition}{}
    La matrice nulle de ${\displaystyle \mathrm {M} _{m,n}\left(K\right)}$, notée 0 ou ${\displaystyle \mathbf {0} _{\mathrm {M} _{m,n}\left(K\right)}}$, est :
${\displaystyle \mathbf {0} _{\mathrm {M} _{m,n}\left(K\right)}:={\begin{pmatrix}0&0&\cdots &0\\0&0&\cdots &0\\\vdots &\vdots &\ddots &\vdots \\0&0&\cdots &0\end{pmatrix}}}{\displaystyle \mathbf {0} _{\mathrm {M} _{m,n}\left(K\right)}:={\begin{pmatrix}0&0&\cdots &0\\0&0&\cdots &0\\\vdots &\vdots &\ddots &\vdots \\0&0&\cdots &0\end{pmatrix}}}$
où 0 est l'élément neutre pour l'addition dans l'anneau K — si$ {\displaystyle K=\mathbb {R}}$ ou ${\displaystyle \mathbb {C}}$, c’est simplement le zéro habituel.
\end{Définition}
\subsubsubsection{Matrice identité}
\begin{Définition}[]{Définition}{}
On appelle matrice identité de taille n la matrice

${\displaystyle \mathrm {I} _{n}={\begin{pmatrix}1&0&\cdots &0\\0&1&\cdots &0\\\vdots &\ &\ddots &\vdots \\0&0&\cdots &1\end{pmatrix}}}{\displaystyle \mathrm {I} _{n}={\begin{pmatrix}1&0&\cdots &0\\0&1&\cdots &0\\\vdots &\ &\ddots &\vdots \\0&0&\cdots &1\end{pmatrix}}}.$
\end{Définition}
\begin{Propriété}[]{Propriétés}{}
    \begin{itemize}
        \item ${\displaystyle \forall A\in \mathrm {M} _{m,n}\left(K\right)\quad A\mathrm {I} _{n}=A}.$
        \item ${\displaystyle \forall B\in \mathrm {M} _{n,p}\left(K\right)\quad \mathrm {I} _{n}B=B}$
    \end{itemize}
\end{Propriété}
\subsubsubsection{Propriétés (binome, produit, triangulaire)}
\subsection{Mn(K)}
\subsubsubsection{Transposition d'une matrice: }
\begin{Définition}[]{Définition}{}
    La matrice transposée ou la transposée d'une matrice ${\displaystyle A\in \mathrm {M} _{m,n}(K)}$ est la matrice notée ${\displaystyle ^{\operatorname {t} }\!A\in \mathrm {M} _{n,m}(K)}$ (aussi  notée ${\displaystyle A^{\operatorname {T} }} $ou ${\displaystyle A^{\operatorname {t} }}$), B telle que:
\begin{center}
${\displaystyle \forall (i,j)\in \{1,\ldots ,n\}\times \{1,\ldots ,m\}\qquad b_{i,j}=a_{j,i}}$.
\end{center}
\end{Définition}
\subsubsubsection{Inverse d'une matrice: }
\begin{Définition}[]{Définition}{}
    Soit ${\displaystyle A}$ une matrice carrée de taille n × n. Lorsqu'elle existe, on appelle inverse à gauche (resp à droite) de ${\displaystyle A}$, une matrice telle que : ${\displaystyle A_{G}^{-1}\,A= {I} _{n}}$.(resp ${\displaystyle A\,A_{D}^{-1}= {I} _{n}}$)
    Lorsqu'une matrice admet un inverse à gauche et à droite on dit que cette matrice est inversible , cet inverse ainsi est unique et l'on note ${\displaystyle A^{-1}}$, 
\end{Définition}
\begin{Théorème}[]{Théorème}{}
    Une matrice carrée ${\displaystyle A}$ est inversible si et seulement si ${\displaystyle \det A\neq 0}$, et dans ce cas, on a :
\begin{center}
${\displaystyle A^{-1}={\frac {1}{\det A}}\,{}^{t}\!\left(\operatorname {com} A\right)}$
\end{center}
\end{Théorème}
\begin{Propriété}[]{Proposition}{}
    Soit ${\displaystyle A\in \mathrm {M} _{n}(K)}$. Les propositions suivantes (dans lesquelles on identifie Mn,1(K) à Kn) sont équivalentes :
\begin{itemize}
    \item[i)] A est inversible.
    \item[ii)]l'application linéaire ${\displaystyle K^{n}\to K^{n},\;X\mapsto AX}$ est bijective (ou, ce qui est équivalent : injective, ou encore : surjective).
    \item[iii)]A est inversible à gauche, c'est-à-dire qu'il existe une matrice B telle que BA = In.
    \item[iv)]A est inversible à droite, c'est-à-dire qu'il existe une matrice B telle que AB = In.
    \item[v)]les colonnes de A forment une base de Kn ;
    la transposée de A est inversible (et dans ce cas, on a ${\displaystyle (^{t}\!A)^{-1}={}^{t}\!(A^{-1})}$.
\end{itemize}


\end{Propriété}
\begin{Propriété}[]{Propriétés}{}
    \begin{itemize}
        \item Si ${\displaystyle \lambda \in K^{*}}{\displaystyle \lambda \in K^{*}},$ la matrice scalaire ${\displaystyle \lambda \mathrm {I} _{n}}$ est inversible : ${\displaystyle (\lambda \mathrm {I} _{n})^{-1}={\frac {1}{\lambda }}\mathrm {I} _{n}}$.
     \item Plus généralement, une matrice diagonale ${\displaystyle \operatorname {diag} \left(\lambda _{1},\dots ,\lambda _{n}\right)}$est inversible si et seulement si tous ses termes diagonaux ${\displaystyle \lambda _{i}}$ sont non nuls, et son inverse est alors ${\displaystyle \operatorname {diag} \left({\frac {1}{\lambda }}_{1},\dots ,{\frac {1}{\lambda }}_{n}\right)}$.
     \item Si une matrice carrée ${\displaystyle A} $est inversible, alors sa transposée l'est aussi, et la transposée de l'inverse de ${\displaystyle A} $est égale à l'inverse de sa transposée :
${\displaystyle ^{\operatorname {t} }\!(A^{-1})=(^{\operatorname {t} }\!A)^{-1}}$.
     \item (n, dim =2 )
    \end{itemize}
\end{Propriété}
\subsubsubsection{Matrice triangulaire: }
\begin{Définition}[]{Définitions}{}
Soit A une matrice de \(M_{n}(K)\)
\\ On appele A triangulaire supérieure si
\(a_{i,j}=0\) pour \(i \ge j\).
\\ On appale A triangulaire inférieure si
\(a_{i,j}=0\) pour \(i \le j\).
\end{Définition}
\subsubsubsection{Matrice diagonale: }
\begin{Définition}[]{Définitions}{}
Soit A une matrice de \(M_{n}(K)\)
\\ On appele A diagonale si
\(a_{i,j}=0\) pour \(i \neq j\).
\end{Définition}
\subsubsubsection{Matrice scalaire: }
\begin{Définition}[]{Définition}{}
Soit A une matrice de \(M_{n}(K)\)

On appelle  A matrice scalaire si  \(\exists \lambda \in K^* \) tel que \(A= \lambda  I_{n}\)
\end{Définition}
\subsubsubsection{Matrices symétriques et antisymétriques: }
\begin{Définition}[]{Définitions}{}
Soit A une matrice de \(M_{n}(K)\),
\begin{itemize}
    \item \\ On appelle A matrice symétrique si 
\\ \(\forall (i,j) \in [|1,n|]^2, a_{i,j}=a_{j,i}\)
\\ (c'est à dire si \(^{\operatorname {t} }\!A=A\)).
\item \\ On appele A matrice antisymétrique 
\\ \(\forall (i,j) \in [|1,n|]^2, a_{i,j}=-a_{j,i}\)
\\ (c'est à dire si \(^{\operatorname {t} }\!A=-A\)).
\end{itemize}


\end{Définition}
\subsubsubsection{Matrice nilpotente:}
\section{Matrice d'une application linéaire} 
Soient E, F, G trois espace vectoriels, avec :
$B=(e_{1},....,e_{m})$ base de E, 
$C=(f_{1},....,f_{n})$ base de F, 
et D base de G.
\subsection{Définition: }
\begin{Définition}[]{Définition}{}
Une application ${\displaystyle u:E\to F}$ est linéaire si et seulement s'il existe une matrice ${\displaystyle A\in \mathrm {M} _{m,n}(K)}$ telle que pour tout vecteur ${\displaystyle x}$ de ${\displaystyle E} $:

Si ${\displaystyle X}$ désigne la matrice colonne des coordonnées de ${\displaystyle x}$ dans la base B
c'est à dire si X=.. avec \(x=\sum i \in [|1,m|] x_{i}e_{i}\) et ${\displaystyle Y}$ celle des coordonnées de ${\displaystyle u(x)}$ dans la base C, c'est à dire si \(Y= \) avec \(u(x)=\sum i \in [|1,n|] y_{i}f_{i}\), alors
\begin{center}
    ${\displaystyle Y=AX}$.
\end{center}

De plus, cette matrice ${\displaystyle A}$ est alors unique : pour tout ${\displaystyle j\in [\![1,n]\!]}$, sa ${\displaystyle j}$-ème colonne est constituée des coordonnées de ${\displaystyle u(e_{j})}$ dans la base C.

La matrice ${\displaystyle A}$ est donc appelée la matrice de ${\displaystyle u}$ dans les bases B, C et notée ${\displaystyle \mathrm {Mat} _{B,C}(u)}$.
\end{Définition}
\paragraph{Remarque: }
Si ${\displaystyle F=E}$ et ${\displaystyle C=B}$, on l'appelle la matrice de ${\displaystyle u} $dans la base B.
\subsection{Théorème: }
\begin{Théorème}[]{Théorème}{}
L'application ${\displaystyle \mathrm {Mat} _{B,C}:\mathrm {L} (E,F)\to \mathrm {M} _{m,n}(K)}$ est un isomorphisme d'espaces vectoriels.
\end{Théorème}
\subsection{Proposition: }
\begin{Propriété}[]{Proposition}{}
Soient ${\displaystyle u:E\to F}$ et ${\displaystyle v:F\to G}$ deux applications linéaires. Alors,

${\displaystyle \mathrm {Mat} _{B,D}(v\circ u)=\mathrm {Mat} _{C,D}(v)\;\mathrm {Mat} _{B,C}(u)}$.
\end{Propriété}
\paragraph{Démonstration: }
\subsection{Corollaire: }
\begin{Propriété}[]{Corollaire}{}
Le produit matriciel est associatif.
\end{Propriété}
\paragraph{Démonstration:}
\section{Changement de bases: }
\subsection{Définition: }
\begin{Définition}[]{Définition}{}
La matrice de passage de ${\displaystyle B} $à ${\displaystyle B'}$ est :

la matrice ${\displaystyle \mathrm {Mat} _{B',B}(\mathrm {Id_{E}} )}$ de l'application identité IdE, de E muni de la base ${\displaystyle B'}$ dans E muni de la base ${\displaystyle B}$
ou, ce qui est équivalent :

la matrice dont les colonnes sont les coordonnées dans ${\displaystyle B} $des vecteurs de ${\displaystyle B'}$.
\end{Définition}
\subsection{Remarque: }
\begin{Définition}[]{Définition}{}
Soit ${\displaystyle P}$ la matrice de passage de ${\displaystyle B}$ à ${\displaystyle B'}$. Il résulte immédiatement de la définition que :

${\displaystyle P} $est inversible : son inverse est la matrice de passage de ${\displaystyle B'}$à ${\displaystyle B} $;
si un même vecteur de E a pour coordonnées ${\displaystyle X}$ dans ${\displaystyle B}$ et ${\displaystyle X'}$ dans ${\displaystyle B'}$, alors ${\displaystyle X=PX'}$.
\end{Définition}
\subsection{Proposition: }
\begin{Propriété}[]{Proposition}{}
Soient :

${\displaystyle u:E\to F}$ une application linéaire ;
${\displaystyle P} $la matrice de passage de ${\displaystyle B}$ à ${\displaystyle B'}$ (bases de ${\displaystyle E}$);
${\displaystyle Q}$ la matrice de passage de ${\displaystyle C}$ à ${\displaystyle C'}$ (bases de ${\displaystyle F}$).
Alors,

${\displaystyle \mathrm {Mat} _{B',C'}(u)=Q^{-1}\;\mathrm {Mat} _{B,C}(u)\;P}$.
\end{Propriété}
\paragraph{Démonstration: }
\section{Rang: }
\subsection{Définition: }
\begin{Définition}[]{Définition}{}
Soit A une matrice de \(M_{(p,q)}(K)\), on appelle rang de A , le rang de ses vecteurs colonnes dans \(K^p\), et on le note rg(A).
Dans le cas oû A est une matrice d'une application linéaire u , on a : rg(A)=rg(u).
\end{Définition}
\begin{Propriété}[]{Propriétés}{}
\begin{itemize}
    \item Si \(A \in M_{p,q}(K), rg(A) \leq inf\)\(\{p,q\}.\)
    \item Si \(A \in M_{n}(K)\) alors 
    \\ (A est inversible.) \(\Leftrightarrow\) (rg(A)=n.)
\end{itemize}
\end{Propriété}
\begin{Théorème}[]{Théorème}{}
Soient \(A \in M_{p,q}(K)\) et \(r \in N^*\), si rg(A)=r alors A est équivalent à \(J_{r}\) avec
\\ \(J_{r}=\begin{matrix}
    I_{r} && 0
    0     && 0
\end{matrix}
\end{Théorème}
\begin{Définition}[]{Définition}{}
Soit \(A=(a_{i,j})_{(i,j) \in [|1,p|]*[|1,q|] } \in M_{(p,q)}(K)\), et soient deux sous-ensembles non vide \(I \subset \{1,...,p\}\) et \(J \subset \{1,...,q\}\).
On appelle la matrice \((a_{i,j})_{(i,j) \in I*J} \in M_{(p,q)}(K)\) matrice extraite et A matrice bordante.
\end{Définition}
\begin{Théorème}[]{Théorème}{}
Soient \(A \in M_{p,q}(K)\), son rang est égale à la taille de la plus grande matrice carrée inversible qu'on peut extraire de cette dernière.
\end{Théorème}
\begin{Propriété}[]{Corollaire}{}
Le rang de la transposée d'une matrice est égal à celui de la dernière.
\end{Propriété}
\section{Equivalence, similitude et trace: }
\subsection{Equivalence:}
\subsubsubsection{Définition: }
\begin{Définition}[]{Définition}{}
Deux matrices ${\displaystyle M}$ et ${\displaystyle N}$ sont dites équivalentes s'il existe deux matrices inversibles ${\displaystyle P}$ et ${\displaystyle Q} $telles que :

${\displaystyle N=Q^{-1}MP}$.
\end{Définition}
\subsubsubsection{Théorème: }
\begin{Théorème}[]{Théorème}{}
Deux matrices de même taille sont équivalentes si et seulement si elles ont même rang.
\end{Théorème}
\paragraph{Démonstration: }
k
\subsection{Similitude: }
\subsubsubsection{Définition: }
\begin{Définition}[]{Définition}{}
Deux matrices carrées ${\displaystyle M}$ et ${\displaystyle N}$ sont dites semblables s'il existe une matrice inversible ${\displaystyle P} $telle que :

${\displaystyle N=P^{-1}MP}$.
\end{Définition}
\subsection{Trace}
\subsubsubsection{Définition: }
\begin{Définition}[]{Définition}{}
Soit A une matrice carrée. La trace de A est la somme des « éléments diagonaux » de A (les éléments de sa diagonale principale). Elle est notée : ${\displaystyle \mathrm {tr} \,\mathbf {A} }$, ou ${\displaystyle \mathrm {Tr} \,\mathbf {A} }$.
\end{Définition}
\subsubsubsection{Propriétés: }
\begin{Propriété}[]{Propriété}{}
L'application \(tr:M_{n}(K) \rightarrow K\) est une forme linéaire.
\end{Propriété}
\begin{Propriété}[]{Propriétés}{}
Soient ${\displaystyle \mathbf {A}}$ et ${\displaystyle \mathbf {B}}$ deux matrices carrées de même taille et ${\displaystyle a}$ un scalaire. Alors :
${\displaystyle \mathrm {tr} \,\left(\mathbf {A} +\mathbf {B} \right)=\mathrm {tr} \,\mathbf {A} +\mathrm {tr} \,\mathbf {B} }$;
${\displaystyle \mathrm {tr} \,\left(a\,\mathbf {A} \right)=a\,\mathrm {tr} \,\mathbf {A} }$;
${\displaystyle \operatorname {tr} (^{\operatorname {t} }\!\mathbf {A} )=\operatorname {tr} \,\mathbf {A} }$;
\end{Propriété}
\paragraph{Démonstration: }
\paragraph{Remarque:}
Si \(A,B,C \in M_{n}(K)\) on peut avoir \(tr(ABC)=tr(CAB)=tr(BCA)\) mais \(tr(ABC) \neq tr(ACB)\).
\begin{Propriété}[]{Corollaire}{}
Soient ${\displaystyle \mathbf {A} \in \mathrm {M} _{m,n}(K)}$ et ${\displaystyle \mathbf {B} \in \mathrm {M} _{n,m}(K)}$. Alors :
${\displaystyle \mathrm {tr} \,\left(\mathbf {AB} \right)=\mathrm {tr} \,\left(\mathbf {BA} \right)}$.
\end{Propriété}
\paragraph{Démonstration: }
m
\begin{Propriété}[]{Propriété}{}
Deux matrices semblables ont même trace.
\end{Propriété}
\begin{Propriété}[]{Propriété}{}
Soit E un K-espace vectoriel et \(p \in L(E)\) un projecteur, alors 
\\ \(tr p = rg(p).1_{K}\)
\end{Propriété}
 \section{Matrice définie par bloc}
\section{Exercices:}


\section{Compléments:}
\subsection{Matrice diagonalement dominantes -HP}
\begin{Définition}[]{Définition}{}
Soit \(A=(a_{i,j})_{1\leq i,j\leq n} \in M_{n}(C)\)
A est dite de diagonale dominante si : 
\\
\begin{center}
\(\forall i \in [|1,n|], \sum_{1 \leq j \leq n ,j\neq i}|a_{i,j}| \le |a_{i,i}|\)
\end{center}
\end{Définition}
\begin{Propriété}[]{Lemme de Hadamard}{}
Soit A une matrice de diagonale dominante, A est inversible.
\end{Propriété}
\paragraph{Démonstration}
\chapter{Déterminants}
\chapter{Compléments: Dualité et \(Gl_{n}(\mathbb{K}\)}
\chapter{PROBLEMES}
\section{Traces:}
\subsection{Matrice de trace nulle}
\begin{Ex}[]{Matrice de trace nulle}{}
Soit \(A \in M_n(\mathbb{R})\) telle que \(TrA=0_{R}\)
\begin{itemize}
    \item[1)] Montrer que A est semblable à une matrice n'ayant que de \(0_{R}\) dans sa diagonale.
    \item[2)] Montrer que \(\exists X,Y \in M_n(\mathbb{R})\) tels que \(A=XY-YX\)
\end{itemize}
\end{Ex}
\subsection{Traces modulo p}
\begin{Ex}[]{Exercice:}{}
Soit \(p\) un nombre premier et \(A,B \in M_{n}(\mathbb{Z})\).
\begin{itemize}
    \item[1)] Montrer que \(Tr((A+B)^p)=Tr(A^p)+Tr(B^p)[p}\).
    \item[2)] En déduire que \(Tr(A^p)=Tr(A)[p]\).
    \item[3)] Soit la suite récurrente \((u_n)\) définie par \(u_0=3, u_1=0, u_2=2\)
    \\ \(\forall n\in \mathbb{N}, u_{n+3}=u_{n+1}+u_{n}\).
    \\Montrer  p divise \(u_{p}\).
\end{itemize}
\end{Ex}
\section{Formule de Burnside, Théorème de Mashke(après matrice hehehehe)}
\section{Dérivation:}
\section{Famille positivement génératrice:}
\section{Décomposition de Fitting}
\begin{Ex}[]{Exercice:}{}
Soient E un K-espace vectoriel de dimension finie \( n \in \N^*\), et \(u \in L(E)\):
\\ \begin{itemize}
    \item[1)] Montrer que les suites \((Im u^k)_{k \in N}\) et \((Ker u^k)_{k \in N}\) sont strictement monotones pour l'inclusion jusqu'un certain rang m d'oû elle deviennent stationnaire.
    \item[2)] Montrer que la suite \((dim_{K}Ker(u^{k+1})-dim_{K}Ker(u^{k}))_{k\geq 0}\) est décroissante.
    \item[3)] Montrer que \(E=ker u^m \oplus Im u^m\).
    \item[4)]
\end{itemize}
\end{Ex}
\paragraph{Correction:}
\section{Identité de Sylvester, Identité de Jacobi}
\begin{Ex}[]{Exercice:}{}
\begin{itemize}
    \item[1)]Identité de Jacobi:
    Soit \(A \in \mathbb{Gl}_n(\mathbb{K})\).
    On note \(T=A^{-1}\) et on considère l'écriture par blocs des matrices A et T:
    \\ A=\begin{pmatrix}
        B && C
        \\S && E
    \end{pmatrix}
    T=\begin{pmatrix}
        W && X
        \\Y && Z
    \end{pmatrix}
     avec \(B,W \in \mathbb{M}_r(\mathbb{K})\) et \(1\leq r \le n \).
    \\ Montrer que \((detA)(detW)=detE\).
    \item[2)] Soient \(I, J \subset \{1,...,n\}, card(I)=card(J)=r, A=(a_{i,j})_{(i,j)\in[|1,n|]^2} \in \mathbb{M}_n(\mathbb{K}) \) On note \(A_{I,J} \in \mathbb{M}_r(\mathbb{K})\) la matrice extraite de A.
    \\ Notons \(I*, J*\) les complémentaires de \(I, J\) das \(\{1,...,n\}\) et \(S(I,J)=\sum_{i\in I}i+\sum_{j \in J}j\).
    \\ Montrer que :
        \\ (\(A \in \mathbb{Gl}_n(\mathbb{C})\) avec \(T=A^{-1}\)) \(\Rightarrow\) (\((detA)(detT_{J,I})=(-1)^{S(I,J)}(detA_{I*,J*})\))
  
\item[3)] Identité de Sylvester:
Soit \(A=(a_{i,j})_{(i,j)\in[|1,n|]^2 \in \mathbb{M}_n(\mathbb{K})\), on pose :
\\ \(\Gamma_{I,J}=det(com(A))_{I,J}, \Delta_{I,J}=det(A_{I*,J*})\).
\\ Montrer que : 
\begin{center}
    \(\Gamma_{I,J}=(-1)^{S(I,J)}.\Delta_{I,J}.(detA)^{r-1}\)
\end{center}*
\end{itemize}
\end{Ex}
\paragraph{Correction:}
\section{Dual de Mn(K)}
\begin{Ex}[]{Exercice:}{}
Si \(A \in \mathbb{M}_n(\mathbb{K})\) , on note \(\phi_{A}\) la forme linéaire définie  par :
\(\phi_{A}:
    \mathbb{M}_n(\mathbb{K}) \rightarrow \(\mathbb{K}
       \\ X \mapsto Tr(AX)\)
        
\\ Montrer que \(\phi:
    \mathbb{M}_n(\mathbb{K}) \rightarrow \(\ \mathbb{M}_n(\mathbb{K})^*
       \\ A \mapsto \phi_{A}\)
        
        est un isomorphisme.

\end{Ex}
\paragraph{Correction:}
\section{Stabilisation du GLn(K)}
\begin{Ex}[]{Exercice}{}
Soit \(A \in M_n(\mathbb{C})\) et \(\phi \in \mathbb{L}(M_n(\mathbb{C})\) tel que : \((A \in \mathbb{Gl}_n(\mathbb{C}))\) \(\Rightarrow\) \((\phi(A)\in \mathbb{Gl}_n(\mathbb{C}))\)
\begin{itemize}
    \item[1)] Montrer que :
    (\(A \in \mathbb{Gl}_n(\mathbb{C})\)) \(\Leftrightarrow\) (\(\exists P \in \mathbb{Gl}_n(\mathbb{C}), \forall \lambda \in \mathbb{C}, P-\lambda.A \in \mathbb{Gl}_n(\mathbb{C})\))
    \item[2)] Montrer que \((\phi(A)\in \mathbb{Gl}_n(\mathbb{C}))\)\(\Rightarrow\) \((A \in \mathbb{Gl}_n(\mathbb{C}))\)
\end{itemize}
\end{Ex}
\paragraph{Correction:}
\section{Intersection des hyperplans avec GLn(K)}
\begin{Ex}[]{Exercice:}{}
Soit \(n \geq 2\), Montrer que tout hyperplan de \(\mathbb{M}_n(\mathbb{K})\) coupe \(\mathbb{Gl}_n(\mathbb{K})\)
c'est à dire :
\\ \begin{center}
    \(H \subset \mathbb{M}_n(\mathbb{K})\) \(\Rightarrow\) \(\mathbb{Gl}_n(\mathbb{K}) \cap H \neq \varnothing\).
\end{center}  
\end{Ex}
\paragraph{Correction:}
\section{Conservation de similitude par passage vers un surcorps}
\begin{Ex}[]{Exercice:}{}
\begin{itemize}
    \item[1)] Soient\(A,B \in M_n(\mathbb{R})\) semblabes sur \(\mathbb{C}\).
\\ Montrer qu'elles sont sembalbes sur \(\mathbb{R}\).
\item[2)] Cas general:
\\ Soient \(\mathbb{K}\) un corps infini et \(\mathbb{L}\) une extension de \(\mathbb{K}\).
\\ Montrer que si \(A,B \in M_n(\mathbb{K})\) semblabes sur \(\mathbb{L}\), alors elles sont sembalbes sur \(\mathbb{K}\).
\end{itemize}

\end{Ex}
\paragraph{Correction:}
\section{Dimension maximale d'un sous-espace vectoriel de \(M_n(K)\) de rang p}
\begin{Ex}[]{Exercice:}{}
Soit \(n \geq 2\) et \(\mathbb{K}\) un corps commutatif infini.
\\ Soit V un sous-espace vectoriel de \(\mathbb{M}_n(\mathbb{K})\), posons \(p=max\{rgA|M\in V\}\) avec \(p\le n\).
\begin{itemize}
    \item[1)] Montrer que V est isomorphe à un sous-espace vectoriel contenant \(J=\begin{pmatrix}
        I_p&&0_{p,n-p}
        \\0_{n-p,p}&&0_{p,p}
    \end{pmatrix}\).
    Dans la suite on suppose que \(J \in V\).
    \item[2)] Montrer que \(si M \in V,\) alors \(\exists (A,B,C) \in \mathbb{M}_p(\mathbb{K})*\mathbb{M}_{p,n-p}(\mathbb{K})*\mathbb{M}_{n-p,p}(\mathbb{K}),\) tel que (M=\begin{pmatrix}
        A&&B
        \\C&&0_{p,p}
    \end{pmatrix}\).
    et \(BC=0_{M_{n-p,n-p}(K)}\)
    \\On notera d'ailleurs pout tout \(M \in V, A=a(M), B=b(M), C=c(M).\)
    \item[3)] Soient \(M \in V\) tel que \(C=0_{M_{n-p,p}(K)\)et E un sev de \(K^p\) tel que \(E=\cup_{N \in V}Im(C) \)
    \\ Montrer que \(E \subset Ker(b(M)).\) 
    \item[4)] notons \(r=dim_{K}E\) et soit \((e_1,...,e_r)\) une base de E complété par \((e_{r+1},...,e_p)\) en une base de \(\mathbb{K}^p\).
    \\Posons l'application suivante: 
    \\ \(\phi:V\rightarrow M_{p}(K)*K^{n-p}^{p-r}*M_{p,n-p}(K)\\ M=\begin{pmatrix}
        A&&B
        \\C&&0_{p,p}
    \end{pmatrix} \mapsto (A,Be_{r+1},...,Be_{p},C)\)
   \\ Montrer alors que \(dim_KV \leq np.\)
    \item[5)] Supposons \(\mathbb{K}=\mathbb{R}\) Montrer que \(dim_KV \leq np.\)
\end{itemize}
\end{Ex}
\paragraph{Correction:}
\section{Décomposition de Bruhat}
\begin{Ex}[]{Exercice:}{}
\end{Ex}
\paragraph{Correction:}
\chapter{Réduction des endomorphismes et matrice carrées}
\section{Généralités:}
\subsection{Elements propres d'un endomorphisme et de matrice carrée}
\subsubsection{Cas d'endomorphisme:}
Soient K un corps, E un K-espace vectoriel et \(u\) un endomorphisme de \(L(E)\).
\begin{Définition}[]{Définition}{}
Soit \(\lambda\in K\), on dit que \(\lambda\) est une valeur propre de \(u\) si \(\exists x \in E \setminus \{0_{E}\}\) tel que : \(u(x)=\lambda x\) ce qui est équivalent aux assetions suivantes:
\begin{itemize}
    \item[i)] \(ker(u-\lambda.Id_{E}) \neq \{0_{E}\}\).
    \item[ii)] \(u-\lambda.Id_{E}\) n'est pas injective.
    \item[iii)] \(u-\lambda.Id_{E}\) n'est pas inversible (Seulement si E est de dimension finie).
\end{itemize}
L'ensemble des valeurs propres d'un endomorphisme s'appelle son spectre et il dépend du corps de base de l'espace vectoriel ainsi on note:
\begin{center}
\begin{framed}
\(Sp_{K}(u)=\{\lambda \in K|\exists x \in E \setminus \{0_{E}\} , u(x)=\lambda.x\}\) 
\end{framed}
\end{center}
\end{Définition}
\begin{Définition}[]{Définition}{}
Soit \(x \in E \setminus \{0_{E}\}\), on dit que x est une valeur propre de \(u\) si \(\exists \lambda \in K\) tel que : \(u(x)=\lambda x\)
\end{Définition}
\paragraph{Remarque:}
\begin{itemize}
    \item Une valeur propre peut être nulle.
    \item Un vecteur propre ne peut jamais être nul.
\end{itemize}
\begin{Définition}[]{Définition}{}
Si \(\lambda\) est une valeur propre, on appelle sous-espace propre : l'ensemble des vecteurs propres associées à cette valeur qu'on note \(E_{\lambda}(u)\) ainsi on a :
\begin{center}
\begin{framed}
    \(E_{\lambda}(u)=ker(u-\lambda.Id_{E})\)
\end{framed}
\end{center}
\end{Définition}
\subsubsection{Cas d'une matrice carrée:}
\begin{Définition}[]{Définitions}{}
Toutes les définitions précédentes s'étend pour les matrices carrées en considèrant, pour une matrice \(A\in \mathrm {M} _{n}(K)\) l'endomorphisme :
${\displaystyle \varphi _{A}:K^{n}\to K^{n},\ X\mapsto AX}.$
c'est à dire :
\begin{itemize}
    \item   \(\lambda \in K\) est une valeur propre de \(u\) si \(\exists X \in K^n \setminus \{0_{K^n}\}\) tel que : \(AX=\lambda X\) ce qui est équivalent à \(ker(A-\lambda.I_{n}) \neq \{0_{K^n}\}\).
    \item \(X \in K^n \setminus \{0_{K^n}\}\) est une valeur propre de \(A\) si \(\exists \lambda \in K\) tel que : \(AX=\lambda X\).
    \item \(E_{\lambda}(A)=ker(A-\lambda.I_{n})\) est le sous-espace propre associé à la valeur propre \(\lambda\).
\end{itemize}
\end{Définition}
\begin{Théorème}[]{Théorème}{}
Soit \(k \in \mathbb{N}^*\) si \(\lambda_{1},...,\lambda_{k}\) des valeurs propres distincts deux à deux d'un endomorphisme u alors les sous-espaces propres \(\ E_{\lambda_{1}},...,E_{\lambda_{k}}\) sont en somme directe.
\end{Théorème}
\paragraph{Démonstration:}
k
\begin{Propriété}[]{Proposition}{}
Si E est un \(\mathbb{K}\)-espace vectoriel de dimension finie \(n \in K^*\) alors si \(u\) est un endomorphisme de E, il admet au plus \(n\) valeurs propres.
\end{Propriété}
\paragraph{Démonstration: }
k
\subsection{Polynome caractéristique}
Depuis maintenant, on considera que E est un espace vectoriel de dimension \(n \in \mathbb{N}^*\)
\begin{Définition}[]{Définition}{}
Soit \(A \in M_{n}(\mathbb{K})\), on appelle polynôme caractéristique de A , le polynôme noté \(\chi_{A}\) définit par :
\\ \begin{center}
\begin{framed}
    \(\chi_{A}(X)=det(A-X.I_{n})\)
\end{framed}
\end{center}
\end{Définition}
\begin{Définition}[]{Définition}{}
Soit \(u \in L(E)\), on appelle polynôme caractéristique de u , le polynôme caractéristique de sa matrice dans une base quelconque de E, et on le note \(\chi_{u}\).
\end{Définition}
\paragraph{Remarque: }
Par convention, le polynôme caractéristique est unitaire.
\begin{Propriété}[]{Proposition}{}
Soit \(A \in M_{n}(\mathbb{K})\), alors \(\chi_{A}=\chi_{A}\)
\end{Propriété}
\begin{Propriété}[]{Proposition}{}
Deux matrices semblables ont le même polynôme caractéristique.
\end{Propriété}
\paragraph{Démonstration:}
k
\begin{Propriété}[]{}{Proposition-coef}
    
\end{Propriété}
\begin{Propriété}[]{}{Proposition-matricetriangulaire}
    
\end{Propriété}
\begin{Propriété}[]{}{Proposition-induit}
    
\end{Propriété}
\begin{Propriété}[]{Proposition}{}
Soit \(u \in L(E)\), on a:
\begin{center}
\\ (\(\lambda\) valeur propre de \(u\)) \(\Leftrightarrow\) (\(\chi_{u}(\lambda)=0_{K}\))
\end{center}
\end{Propriété}
\begin{Définition}[]{Définition}{}
Soit \(u \in L(E)\) et \(\lambda \in Sp_{K}(u)\), on appelle multiplicité de \(\lambda\) sa multiplicité en tant que racine du polynôme caractéristique de \(u\) et on la note par \(m_{\lambda}\).
\end{Définition}
\begin{Propriété}[]{Proposition}{}
Soit \(u \in L(E)\) et \(\lambda \in Sp_{K}(u)\) on a :
\begin{center}
    \(dim_{K}(E_{\lambda}) \leq m_{\lambda}\).
\end{center}

\end{Propriété}
\subsection{Diagonalisation}
\begin{Définition}[]{Définition}{}
Soit \(u \in L(E)\) , u est diagonalisable s'il existe une base de E composée seulement de vecteurs propres de \(u\).
\\ Dans une telle base, sa matrice est diagonale d'oû la définition suivante:
\\ Un endomorphisme est diagonalisable si sa matrice est semblable à une matrice diagonale.
\end{Définition}
\paragraph{Démonstration: }
k
\begin{Définition}[]{Définition}{}
Une matrice \(A \in M_{n}(K)\) est diagonalisable si elle est semblable à une matrice diagonale c'est à dire si :
\begin{center}
\begin{framed}
\(\exists P \in Gl_{n}(K),   \exists \Lambda =diag(\lambda_{1},...,\lambda{n}) \in M_{n}(K), A=P\Lambda P^{-1}.\)
\end{framed}
\end{center}
\end{Définition}
\begin{Propriété}[]{Conditions de diagonalisabilité}{}
Soit \(u \in L(E)\), avec \(r=card(Sp_{K}(u)), Sp_{K}(u)=\{\lambda_{1},...,\lambda_{r}\}\).
Conditions nécessaires de diagonalisabilité: 
\\ Les assertions suivantes sont équivalentes:
\begin{itemize}
    \item[i)] \(u\) est diagonalisable.    \item[ii)] \(\chi_{u}\) est scindé et  \(\forall i \in [|1,r|]\lambda_{i}\), \(m_{\lambda_{i}}=dim_{K}(E_{\lambda_{i}}\).
    \item[iii)] \(E=\bigoplus_{i \in [|1,r|]} E_{\lambda_{i}}\)
\end{itemize}
Conditions suffisantes de diagonalisabilité:
\\ Les assertions suivantes sont équivalentes:
\begin{itemize}
    \item[i)] \(r=dim_{K}(E)\).  \item[ii)] \(\chi_{u}\) est scindé et à racines simples.
    \item[iii)] les valeurs propres de \(u\) sont distinctes deux à deux.
\end{itemize}
Dans le cas oû une des assertions ci-dessus est verifié alors \(u\) est diagonalisable.
\end{Propriété}
\paragraph{Démonstration:}
k
\begin{Propriété}[]{Proposition}{}
Si \(u \in \mathbb{L}(E)\) est diagonalisable et $F$ un sous-espace vectoriel de $E$ stable par $u$ alors \(u_{|F}\) est diagonalisable.
\end{Propriété}
\subsection{Trigonalisation}
\begin{Définition}[]{Définition}{}
Soit \(u \in L(E)\) , $u$ est trigonalisable  s'il existe une base dans laquelle sa matrice est triangulaire supérieure.
\end{Définition}
\begin{Définition}[]{Définition}{}
Soit \(A \in M_{n}(K)\), $A$ est trigonalisable si elle est semblable à une matrice triangulaire supérieure c'est à dire si:
\\
\end{Définition}
\begin{Propriété}[]{Proposition}{}
Soit \(u \in L(E)\)(resp \(A \in M_{n}(K)\)), u (resp A) est diagonalisable si \(\chi_{u}\)(resp \(\chi_{A}\)) est scindé sur $K$.
\end{Propriété}
\paragraph{Démonstration:}
k
\subsection{Réduction simultanée-HP}
\section{Polynome d'endomorphisme, et de matrice carrée}
\subsection{Généralités}
\begin{Définition}[]{Définition}{}
Soient \(u \in \mathbb{L}(E)\),\(A \in M_n(\mathbb{K})\)
les applications \(f_{u}:K[X] \rightarrow L(E) \ P \mapsto P(u)\), \(f_{A}:K[X] \rightarrow M_n(K)\ P \mapsto P(A)  \) sont des morphismes d'algèbres tels que: \begin{center}
\begin{framed}\\
\(\forall P \in K[X]\)  \(f_{u}(P)=P(u)=\) et \(f_{A}(P)=P(A)=\) \end{framed}\end{center}
Si \(\exists r \in \mathbb{N}^*\) et \(\exists (a_{i})_{i \in [|0,r|]} \in K^r\) tel que \(P(X)=\sum_{i \in [|1,r|]} a_{i}.X^i\) alors \begin{center} \begin{framed}
\(P=\sum_{i \in [|1,r|]}a_{i}.u^i\), \(P=\sum_{i \in [|1,r|]}a_{i}.A^i\)
 \end{framed}
\end{center}
on a donc:
 \(\forall x \in E, f_{u}(P)(x)=\sum_{i \in [|1,r|]}a_{i}.u^i(x)\)
\end{Définition}
\paragraph{Remarque:}
\begin{itemize}
    \item Notons que pour \(i \in \mathbb{N}^*, u^i=u\circ u \circ ... \circ u\) (i fois).
    \item \(P(u) \in L(E)\), ie: P(u) est un endomorphisme!!
    \item \(P(A) \in M_n(K)\), ie: P(A) est une matrice!!
    \item si\(\u \in \mathbb{L}(E), \forall P,Q \in \mathbb{K}[X]\) on a \(P(u) \circ Q(u) = (PQ)(u)\).
\end{itemize}
\begin{Propriété}[]{Proposition}{}
\begin{itemize}
    \item[i)] \(Kerf_{u}=\{P \in K[X]|P(u)=0_{L(E)}\}\) est un idéal de \(K[X]\).
    \item[ii)] \(Imf_{u}=\{P(u)|P \in K[X]\}\) est une sous algèbre commutative de \(L(E)\) noté \(K[u]\).
\end{itemize}
\end{Propriété}
\paragraph{Démonstration:}
.
\subsection{Polynôme minimal}
\begin{Définition}[]{Définition}{}
On considère l'ideal \(\mathbb{I}=Kerf_{u}=\{P \in K[X]|P(u)=0_{L(E)}\}\), on a \(\mathbb{I} \neq \{0_{K[X]}\}\) alors il est généré par un seul élément noté  "\(\mu_{u}\)" ou "\(\pi_{u}\)" c'est à dire:

\begin{center}
\begin{framed}
    \(\mathbb{I}=\(\pi_{u}\)K[X]\)
\end{framed}
\end{center}
 On a donc : 
\begin{center}
\begin{framed}
\\ \((\forall P \in K[X], P(u)=0.) \Leftrightarrow (\(\pi_{u}\)| P.) \)
\end{framed}
\end{center}

\end{Définition}
\paragraph{Démonstration:}
\begin{itemize}
    \item 
    \item
\end{itemize}
\paragraph{Remarque:}
Le polynôme minimal est unitaire.
\begin{Définition}[]{Définition}{}
De même on considère l'ideal \(\mathbb{I}=Kerf_{A}=\{P \in K[X]|P(A)=0_{M_n(\mathbb{K}}\}\), on a \(\mathbb{I} \neq \{0_{K[X]}\}\) alors il est généré par un seul élément noté  "\(\mu_{A}\)" ou "\(\pi_{A}\)" c'est à dire:

\begin{center}
\begin{framed}
    \(\mathbb{I}=\(\pi_{A}\)K[X]\)
\end{framed}
\end{center}
 On a donc : 
\begin{center}
\begin{framed}
\\ \((\forall P \in K[X], P(A)=0.) \Leftrightarrow (\(\pi_{A}\)| P.) \)
\end{framed}
\end{center}
\end{Définition}
\begin{Propriété}[]{Proposition}{}
Si d est le degré du Polynôme minimal d'un endomorphisme alors la famille \((u^k)_{1 \leq k \leq d-1}\) est une base de \(K[u]\).
\end{Propriété}
\begin{Propriété}[]{Proposition}{}
Soit \( u \in L(E), \lambda \in K\) on a :
\\  Les racines de \(\mu_{\lambda}\) dans \(\mathbb{K}\) sont les valeurs propres de \(u\).
ce qui est equivalent à 
\begin{center}
    \((\mu_{\lambda}(\lambda)=0_{K}) \Leftrightarrow (\lambda \in Sp_{K}(u).)\)
\end{center}

\end{Propriété}
\begin{Propriété}[]{Lemme de décomposition de noyaux}{}
Soient \(P_{1},...,P_{r}\) des polynômes premiers entre eux deux à deux tel que \(P=\Pi_{i=1}^rPi\) alors:

\begin{center}
\begin{framed}
    \(Ker(P(u))=\bigoplus_{i=1}^r Ker(P_{i}(u))\).
    \end{framed}
\end{center}
\end{Propriété}
\paragraph{Démonstration:}
k
\begin{Propriété}[]{Proposition}{}
Soit \(u \in L(E)\), les assertions suivantes sont équivalentes:
\begin{itemize}
    \item[i)] \(u\) est diagonalisable.
    \item[ii)] \(\mu_{u}\) est scindé dans \(\mathbb{K}\) à racines simples.
    \item[iii)] \(\exists P \in K[X]\) scindé à racines simples tel que \(P(u)=0_{L(E)}\).
\end{itemize}
\end{Propriété}
\begin{Propriété}[]{Proposition}{}
De même si \(A \in M_n{K}\), les assertions suivantes sont équivalentes:
\begin{itemize}
    \item[i)] \(A\) est diagonalisable.
    \item[ii)] \(\mu_{A}\) est scindé dans \(\mathbb{K}\) à racines simples.
    \item[iii)] \(\exists P \in K[X]\) scindé à racines simples tel que \(P(A)=0_{M_n{K}}\).
\end{itemize}
\end{Propriété}
\paragraph{Démonstration:}
k
\begin{Propriété}[]{Proposition}{}
Si \(F\) est un sous-espace vectoriel de E, stable par \(u\) et \(u_{|F}\) l'endomorphisme induit par \(u\) sur \(F\) alors \(\mu_{u_{|F}}\) divise \(\mu_{u}\).
\\ Et si \(u\) est diagonalisable alors \(u_{|F}\) est aussi diagonalisable. 
\end{Propriété}
\begin{Propriété}[]{Proposition}{}
Soit \(u \in L(E)\), les assertions suivantes sont équivalentes:
\begin{itemize}
    \item[i)] \(u\) est trigonalisable.
    \item[ii)] \(\mu_{u}\) est scindé dans \(\mathbb{K}\).
    \item[iii)] \(\exists P \in K[X]\) scindé tel que \(P(u)=0_{L(E)}\).
\end{itemize}
\end{Propriété}
\begin{Propriété}[]{Proposition}{}
De même si \(A \in M_n(\mathbb{K}\), les assertions suivantes sont équivalentes:
\begin{itemize}
    \item[i)] \(A\) est trigonalisable.
    \item[ii)] \(\mu_{A}\) est scindé dans \(\mathbb{K}\).
    \item[iii)] \(\exists P \in K[X]\) scindé tel que \(P(u)=0_{M_n{K}}\).
\end{itemize}
\end{Propriété}
\subsection{Théorème de Cayley-Hamilton}
\begin{Théorème}[]{Théorème}{}
Soit \(u \in L(E), \chi_{u}\) son polynôme caractéristique, alors \(\chi_{u}=0_{L(E)}\)
\end{Théorème}
\paragraph{Démonstration:}
k
\subsection{Sous-espace caractéristiques}
\begin{Définition}[]{Définition}
{}
Soit \(u \in \mathbb{L}(E)\) tel que son polynôme caractéristique \(\chi_{u}\) est scindé dans \(\mathbb{K}\) ie : \(\exists s \in \mathbb{N}^*\) tel que \(\exists (\lambda_{1},...,\lambda_{s}) \in \mathbb{K}^s, (\alpha_{1},...,\alpha_{s}) \in \mathbb{N}^s, \\ P=(X-\lambda_{1})^\alpha_{1}...(X-\lambda_{s})^\alpha_{s}\) \\
Pour tout \(i \in [|1,s|]\) on appelle \mathbf{sous-espace caractéristique} le sous-espace vectoriel \(Ker(u-\lambda_{i}.Id)^\alpha_{i}\).
\end{Définition}
\begin{Propriété}[]{Proposition}{}
\begin{itemize}
    \item[i)] Pour tout \(i \in [|1,s|]\), \(Ker(u-\lambda_{i}.Id)^\alpha_{i}\) est stable par \(u\).
    \item[ii)] \(E=\bigoplus_{i=1}^s Ker(u-\lambda_{i}.Id)^\alpha_{i} \).
    \item[iii)] Pour tout \(i \in [|1,s|], dim_{K}(Ker(u-\lambda_{i}.Id)^\alpha_{i})=\alpha_{i}\).
\end{itemize}
\end{Propriété}
\paragraph{Démonstration:}
k
\section{Exercices}
\subsection{Techniques de Diagonalisation}
\section{Compléments}
\subsection{Matrice circulantes}
\subsubsection{Généralités}
\begin{Définition}[]{Définition}{}
Soit \(M \in M_{n}(\mathbb{C})\), on dit que M est une matrice circulante si elle s'écrit sous la forme suivante :
\(M=\begin{pmatrix}
    c_{1}&&c_{2}&&...&&c_{n}
    \\c_{n}&&c_{1}&&...&&c_{n-1}
    \\c_{n-1}&&c_{n}&&...&&c_{n-2}
    \\ \vdots &&\ddots&& \vdots
    \\c_{2}&&c_{3}&&...&&c_{1}
\end{pmatrix}\)
\\avec \(\forall i \in [|1,n|]\, c_{i} \in \mathbb{C} ) 
\end{Définition}
\begin{Propriété}[]{Proposition}{}
On note une matrice circulante de sorte que la matrice précédente est noté \(M(c_1,...,c_n)\)
\\On pose alors \(J=\begin{pmatrix}
    0&&1&&0&&\dots&&0
    \\0&&0&&1&&\dots&&0
    \\0&&0&&0&&\ddots&&0
    \\ \vdots &&\vdots &&\vdots&& \vdots&&\vdots
    \\0&&0&&0&&\dots&&1
    \\1&&0&&0&&\dots&&0
\end{pmatrix}\)
de sorte que 
\\ \(J=M(0,1,...,0)\), on a donc : \\ \begin{center}(M est une matrice circulante) \(\Leftrightarrow\) (M est un polynôme en J)
\end{center}
\end{Propriété}
\paragraph{Démonstration:}
k
\begin{Propriété}[]{Proposition}{}
L'ensemble des matrice circulantes est une sous-algébre commutative de \(M_n(\mathbb{C})\)
\end{Propriété}
\paragraph{Démonstration:}
k
\subsubsection{Réduction des matrices circulantes}
\begin{Propriété}[]{Réduction de la matrice J}{}
La matrice J est diagonalisable, ses valeurs propres sont les racines n-ièmes de l'unité et ses vecteurs propres s'expriment ainsi:
\\ On pose \(w=exp(2i\pi/n)\) alors pour tout \(k \in [|1,n|], w^k\) est valeur propre de J et :
\(\forall k\in [|1,n|],\X=\begin{pmatrix}
    1&&
    \\w^k&&
    \\w^{2k}&&
    \\\vdots&&
    \\w^{(n-1)k}&&
\end{pmatrix}\)
est vecteur propre de J.
\end{Propriété}
\paragraph{Démonstration}
On a \(J^n=I_{n}\) donc le polynôme \(X^n-1\) est annulateur de J, ce dernier étant scindé a racines simples dans \(\mathbb{C}\), J est diagonalisable.
\\ Les valeurs propres de J sont les racines de \(X^n-1\), du coup elles sont les racines n-ièmes de l'unité 
\begin{Propriété}[]{Réduction d'une matrice circulante}{}
    
\end{Propriété}
\subsection{Matrice de Toeplitz}
\subsubsection{Définition}
\begin{Définition}[]{Définition}{}

\end{Définition}
\subsubsection{Matrice de Toeplitz tridiagonale}
\begin{Définition}[]{Définition}{}

\end{Définition}
\begin{Propriété}[]{Valeurs propres}{}
    
\end{Propriété}
\paragraph{Démonstration}
k
\subsection{Matrice de Hankel}
\begin{Définition}[]{Définition}{}

\end{Définition}
\begin{Propriété}[]{Valeurs propres}{}
    
\end{Propriété}
\subsection{Matrice monotones }
\begin{Définition}[]{Définition}{}

\end{Définition}
\subsection{Matrices de transvections et de dilatation}
\begin{Définition}[]{Définition}{}

\end{Définition}
\
\subsection{Dunford}
\subsection{Jordan}
\subsection{Frobenius}
\subsection{Simplicité}
\subsection{Nilpotence}
\subsection{Stochastique}
\end{document}
